% sezione intro frame 01
\begin{frame}
    \frametitle{Introduzione al Corso di Codifica dei Testi}
    \framesubtitle{Obiettivi, competenze e conoscenze}
    \addtocounter{nframe}{1}
    
    \begin{block}{Obiettivo}
        L'obiettivo del corso è quello di illustrare i principi di modellazione e le prassi di codifica del testo per una opportuna rappresentazione ed elaborazione digitale di risorse testuali.  
    \end{block}

    \begin{block}{Rationale}
       Il corso si propone di fornire agli studenti gli strumenti e la metodologia necessaria per effettuare la codifica digitale di testi, in particolar modo di testi letterari o comunque di interesse storico-culturale, usando gli schemi TEI XML.
    \end{block}

\end{frame}

% sezione intro frame 02
\begin{frame}
    \frametitle{Argomenti trattati}
    \framesubtitle{Obiettivi, competenze e conoscenze}
    \addtocounter{nframe}{1}
    
    \begin{block}{Competenze attese}
        Al termine del corso sarete in grado di valutare il metodo di codifica più appropriato allo scenario d'interesse, di creare uno schema di codifica TEI e di usare gli strumenti più idonei per la codifica e la (semplice) elaborazione e visualizzazione di un testo.
    \end{block}

    

\end{frame}

% sezione intro frame 03
\begin{frame}
    \frametitle{Principali Argomenti}
    \framesubtitle{Obiettivi, competenze e conoscenze}
    \addtocounter{nframe}{1}

    
        \begin{itemize}
            \item Codifica dei caratteri e di testi
            \item I linguaggi di markup e introduzione a XML
            \item Creazione di schemi di codifica
            \item Le norme TEI (Text Encoding Initiative)
            \item Alcuni specifici Moduli TEI
            \item Definizione di schemi di codifica personalizzati
            \item introduzione ai fogli di stile XSLT
            \item elaborazione documenti XML-TEI (XSLT2.0, DOM)
            \item esempi, esercitazioni e seminari 
        \end{itemize}
    

\end{frame}