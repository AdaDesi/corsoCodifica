\begin{frame}
    \frametitle{Elementi XML}
    \framesubtitle{Conclusioni}
    \addtocounter{nframe}{1}

    \begin{block}{XML per rappresentare il testo}
    %\begin{center}\texttt{<!DOCTYPE root PUBLIC “id” “uri”>}\end{center}
    %\begin{center}\texttt{standard//owner//description//language}\end{center}
    %\begin{center}\texttt{-//W3C//DTD XHTML 1.0 Strict//EN}\end{center}
        \begin{itemize}
            \item I markup language per supportare la rappresentazione, memorizzazione, pubblicazione di un testo.
            \item XML è un markup language flessibile e potente.
        \end{itemize}

        \begin{itemize}
            \item le istruznioni dei markup language sono per lo più dichiarazioni indicando particolari funzioni del dato.
            \item le istruznioni sono etichette visibili.
        \end{itemize}
        
    \end{block}

\end{frame}

\begin{frame}
    \frametitle{Elementi XML}
    \framesubtitle{Conclusioni}
    \addtocounter{nframe}{1}

    \begin{block}{XML per rappresentare il testo}
    %\begin{center}\texttt{<!DOCTYPE root PUBLIC “id” “uri”>}\end{center}
    %\begin{center}\texttt{standard//owner//description//language}\end{center}
    %\begin{center}\texttt{-//W3C//DTD XHTML 1.0 Strict//EN}\end{center}
        \begin{itemize}
            \item Una sintassi e una grammatica regolano l'applicabilità del linguaggio di marcatura
            \item Sintassi: documento well formed (ben formato)
            \item Grammatica: documento valido
        \end{itemize}

    \end{block}

\end{frame}


\begin{frame}
    \frametitle{Elementi XML}
    \framesubtitle{Conclusioni}
    \addtocounter{nframe}{1}

    \begin{block}{XML per rappresentare il testo}
    %\begin{center}\texttt{<!DOCTYPE root PUBLIC “id” “uri”>}\end{center}
    %\begin{center}\texttt{standard//owner//description//language}\end{center}
    %\begin{center}\texttt{-//W3C//DTD XHTML 1.0 Strict//EN}\end{center}
        \begin{itemize}
            \item XML deriva dal linguaggio SGML.
            \item XML è una specifica del consorzio W3C.
            \item XML è un meta-linguaggio.
            \item XML è plain text.
            \item XML è portabile.
        \end{itemize}

    \end{block}

\end{frame}

\begin{frame}
    \frametitle{Elementi XML}
    \framesubtitle{Conclusioni}
    \addtocounter{nframe}{1}

    \begin{block}{XML per rappresentare il testo}
    %\begin{center}\texttt{<!DOCTYPE root PUBLIC “id” “uri”>}\end{center}
    %\begin{center}\texttt{standard//owner//description//language}\end{center}
    %\begin{center}\texttt{-//W3C//DTD XHTML 1.0 Strict//EN}\end{center}
        \begin{itemize}
            \item XML definisce markup dichiarativi e descrittivi.
            \item XML ha un modello dati ad albero ordinato.
            \item XML può avere associato un tipo di documento (DTD) o uno schema (XSD).
        \end{itemize}

    \end{block}

\end{frame}
