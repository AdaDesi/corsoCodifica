%slide relative ad introdurre gli elementi di XML

% prendere da libro XML in amazon e XML visual quick view
% inserire qualche nota sul namespace anche preso dal libro xsd da pag 26 
% riprendere qualche slide dall slide Del Turco; e dalle slide Fiormonte-Ciotti-Silvi-XML-corso-2014.pdf
% http://filologiadigitale-verona.it/wp-content/uploads/2014/10/Fiormonte-Ciotti-Silvi-XML-corso-2014.pdf
% Slide Chiara di Pietro

% IDE/Editor Management Studio comes with a nice XML/XSD editor. It has a number of features that makes schema writing lot easier. Intellisense, auto-completion, real-time syntax checks, etc., are a few of those features.

XML was originally created to structure, store, and transport information

XML has its roots in Standard Generalized Markup Language (SGML), a language
introduced in the 1980s that describes the structure and content of any machine-
readable information.

XML can be
thought of as a lightweight version of SGML

Like SGML, XML is a language used to
create vocabularies for other markup languages

XML is a markup language that is extensible, so it can be m­odified
to match the needs of the document author and the data being recorded

he ­standards
for XML are developed and maintained by the World Wide Web Consortium (W3C),
an organization created in 1994 to develop common protocols and standards for
sharing information on the World Wide Web

XML, or eXtensible Markup Language, is a
specification for storing information. It is also
a specification for describing the structure of
that information. And while XML is a markup
language (just like HTML), XML has no tags
of its own

The syntax rules of XML are easy
to learn and easy to use


Every XML element must have a closing tag. Every element must have a closing tag. A self-closing tag is
permitted.

XML tags are case sensitive. Opening and closing tags (or start and end tags) must be ­written
with the same case.

XML elements must be ­properly nested. All elements can have child (sub) elements. Child elements must
be in pairs and be correctly nested within their respective parent
element.

Every XML document must have a root element. Every XML document must contain a single tag pair that defines
the root element. All other elements must be nested within the
root element.

XML elements can have attributes in name-value pairs. Each attribute name within the same element can occur only once.
Each attribute value must be quoted.

Some characters have a ­special meaning in XML. The use of certain characters is restricted. If these characters are
needed, entity references or character references may be used.
References always begin with the character “&” (which is ­specially
reserved) and end with the character “;”.

XML allows for comments. Comments cannot occur prior to the XML Declaration. Comments
cannot be nested.

XML is based on hierarchical trees in
which order is significant 
In XML, hierarchy and sequence are the main methods
used to represent information.

Both HTML and XML use tags in
similar ways, often creating distinctly hierarchical structures to present data to users.


Like SGML,
XML can be used to create XML applications or vocabularies, which are markup
languages tailored to contain specific pieces of information.

Like HTML documents, XML documents can be created and viewed with a basic text
editor such as Notepad or TextEdit. More sophisticated XML editors are available, and
using them can make it easier to design and test documents.


You can think of a standard vocabulary as a set of XML tags for a particular
industry or business function. As XML has grown in popularity, standard vocabularies
continue to be developed across a wide range of disciplines.


To meet the need of textual scholars, an XML ­vocabulary
called Text Encoding Initiative (TEI) was developed, which codes text
i­nformation.

SEMPLICE ESEMPIO XML TEI


tanti vocabolari XML
Bioinformatic Sequence Markup
Language (BSML) Coding of bioinformatic data
Extensible Hypertext Markup Language
(XHTML) HTML written as an XML application
Mathematical Markup Language
(MathML) Presentation and evaluation of mathematical equations
and operations
Music Markup Language (MML) Display and organization of music notation and lyrics
Weather Observation Definition
Format (OMF) Distribution of weather observation reports, forecasts, and
advisories
Really Simple Syndication (RSS) Distribution of news headlines and syndicated columns
Synchronized Multimedia Integration
Language (SMIL) Editing of interactive audiovisual presentations ­involving
streaming audio, video, text, and any other media type
Voice Extensible Markup Language
(VoiceXML) Creation of audio dialogues that feature synthesized
speech, digitized audio, and speech recognition
Wireless Markup Language (WML) Coding of information for smaller-screened devices, such
as PDAs and cell phones


For different users to share a vocabulary effectively, rules must be developed that
specifically control what code and content a document from that vocabulary might
contain. This is done by attaching either a Document Type Definition (DTD) or a
schema to the XML document containing the data. Both DTDs and schemas contain
rules for how data in a document vocabulary should be structured. A DTD defines the
structure of the data and, very broadly, the types of data allowable. A schema more
precisely defines the structure of the data and specific data restrictions.

To ensure a document’s compliance with XML rules, it can be tested against two
standards—whether it’s well formed, and whether it’s valid. A well-formed document
contains no syntax errors and satisfies the general specifications for XML code as laid
out by the W3C. At a minimum, an XML document must be well formed or it will not
be readable by programs that process XML code.

If an XML document is part of a vocabulary with a defined DTD or schema, it also
must be tested to ensure that it satisfies the rules of that vocabulary. A well-formed XML
document that satisfies the rules of a DTD or schema is said to be a valid document.

Because XML focuses on
communicating the data, the overall structure is simple and easy to design and
maintain.

An XML document consists of three parts—the prolog, the document body, and the
epilog. The prolog includes the following parts:
• XML declaration: indicates that the document is written in the XML language
• Processing instructions (optional): provide additional instructions to be run by
programs that read the XML document
• Comment lines (optional): provide additional information about the document
contents
• Document type declaration (DTD) (optional): provides information about the rules
used in the XML document’s vocabulary

ESEMPIO CUSTOM


The document body, found immediately after the prolog, contains the document’s
c ­ ontent in a hierarchical tree structure. Following the document body is an optional
e­pilog, which contains any final comments or processing instructions.

Il prologo di XML: XML declaration 
<?xml version=”version number” encoding=”encoding type”
standalone=”yes|no” ?>

Creating an XML Declaration
• To create an XML declaration, enter the code
<?xml ?>
in the first line of an XML document.
• To specify a version of XML to use, enter the code
version=”version number”
after the opening <?xml tag, where version number is either 1.0 or 1.1.
• To specify a character encoding, enter the code
encoding=”encoding type”
after the version attribute-value pair, where encoding type identifies the ­character
set used in the document.
• To indicate whether the document is a standalone document, enter the code
standalone=”yes|no”
after the encoding attribute-value pair, where the value yes or no ­indicates whether
access to external files will be needed when processing the document.


ERRORI:
<?XML VERSION=”1.0” ENCODING=”ISO-8859-1” STANDALONE=”YES” ?>
<?xml version=1.0 encoding=ISO-8859-1 standalone=yes ?>
<?xml version=”1.0” standalone=”yes” encoding=”ISO-8859-1” ?>

Generally speaking, comments are ignored by programs reading the document and
do not affect the document’s contents or structure.
To insert a comment in an XML document, enter
<!-- comment -->

If you have a comment that will occupy more than one line, you can
continue the ­comment on as many lines as you need


ESERCIZIO

<!--
This document contains data on SJB Pet Boutique
holiday specials
Filename: project.xml
Author:
your name
Date:
today's date
-->


A program that reads and interprets an XML document is called an XML processor or XML parser, or simply a processor or parser.

First well-formed
A second function of a parser is to interpret PCDATA in a document and resolve any character or
entity references found within the document
finally, an XML document might contain
processing instructions that tell a parser exactly how the document should be read and
interpreted

The current versions of all major web browsers include an XML parser of some type.

To test for well-formedness, you’ll use an XML parser to compare the XML document against the
rules established by the W3C.
INTRODURRE XMLLINT


The document body in an XML document is made up of elements that contain data

to be stored in the document. Elements are the basic building blocks of XML files.

An ­element can have text content and child element content.

<element>content</element>
The opening tag is <element> , and </element> is the closing
tag.

Element names might be established already if an author is using a particular XML
vocabulary, such as TEI-XML


There are a few important points to remember about XML elements:
• Element names are case sensitive, which means that, for example, itemnumber,
itemNumber, and ItemNumber are unique elements.
• Element names must begin with a letter or the underscore character ( _ ) and may not
contain blank spaces. Thus, you cannot name an element Item Number , but you can
name it Item_Number .
• Element names cannot begin with the string xml because that group of ­characters is
reserved for special XML commands.
• The name in an element’s closing tag must exactly match the name in the ­opening tag.
• Element names can be used more than once, so the element names can mean
different things at different points in the hierarchy of an XML document.

Creating XML Elements
XML 24
• To create an XML element, use the syntax
<element>content</element>
where element is the name given to the element, content represents the text
content of the element, <element> is the opening tag, and <
­ /element> is the
closing tag.
• To create an empty XML element with a single tag, use the following syntax:
<element />
• To create an empty XML element with a pair of tags, use the syntax
<element></element>

% XML document is primarily composed of elements and attributes
% Attributes can only exist within an element. An attribute declaration really does not make sense without an element.
% Attributes can only store a value, while elements can also store child elements and attributes.
% An element can appear more than once within a parent node, but an attribute can appear only once. The order of attributes is not significant and there is no way to control the order of attributes in XSD. If the attribute is present the default value is not assigned, even if the value of the attribute is an empty string.
% the default value of elements is assigned only when the element is present and is empty
% Elements are mandatory by default while attributes are optional by default.

% slide su XML lint

