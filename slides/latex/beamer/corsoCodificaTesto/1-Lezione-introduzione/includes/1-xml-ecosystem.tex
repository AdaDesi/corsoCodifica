






% sezione XML frame 01
\begin{frame}
	\frametitle{XML titolo}
	\framesubtitle{XML sottotitolo}
	\addtocounter{nframe}{1}

	\begin{block}{block titolo}
		La riflessione sui metodi e le pratiche migliori per la codifica elettronica dei testi è stata uno dei temi fondamentali della ricerca e della sperimentazione nel dominio dell’Informatica umanistica per molti anni.
	\end{block}

\end{frame}

% sezione XML frame 02
\begin{frame}
	\frametitle{XML titolo}
	\framesubtitle{XML sottotitolo}
	\addtocounter{nframe}{1}

	\begin{block}{block titolo}
		Ad oggi la soluzione considerata ottimale per una corretta rappresentazione del testo è l'adozione dei markup language descrittivi basati su XML.
	\end{block}

	\begin{block}{block titolo}
		Standard de facto per la codifica dei testi è considerato ad oggi lo schema messo a punto dalla Text Encoding Initiative (TEI).
	\end{block}

\end{frame}


% sezione XML frame 03
\begin{frame}
	\frametitle{XML titolo}
	\framesubtitle{XML sottotitolo}
	\addtocounter{nframe}{1}

	\begin{block}{Perché XML}
		importanza dell'XML
		%% leggere capitolo 1 del libro XML schema complete reference 2003 (cliff, etc..)
		importanza della definizione di uno schema XML
		importanza della definizione di trasformate XSL e manipolazione del DOM
	\end{block}
	[SLIDE DA COMPLETARE]

\end{frame}


% sezione XML frame 04
\begin{frame}
	\frametitle{XML titolo}
	\framesubtitle{XML sottotitolo}
	\addtocounter{nframe}{1}

	\begin{itemize}
		\item XSD: XML Schema Definition Language
		\item XPath: XML Path Language
		\item XSL: eXtensible Stylesheet Language
		\item XSL-T:  XSL – Transformations
		\item XSL-FO: XSL – Formatting Objects
		\item XQuery: XML Query Language for XML Databases
		\item XInclude: XML inclusion Language
		\item DTD: Document Type Definition Language
		\item RelaxNG: Regular Expression Language for XML (New Generation)
	\end{itemize}

\end{frame}


% sezione XML frame 05
\begin{frame}
	\frametitle{XML titolo}
	\framesubtitle{XML sottotitolo}
	\addtocounter{nframe}{1}

	\begin{block}{Perché XML}
		Adottando la tecnologia e il linguaggio XML abbiamo la possibilità di creare linguaggi di marcatura personalizzati e specifici per ogni esigenza e dominio.
	\end{block}

\end{frame}

\begin{frame}
	\frametitle{XML titolo}
	\framesubtitle{XML sottotitolo}
	\addtocounter{nframe}{1}

	\begin{block}{Vantaggi}
		Attraverso XML è possibile strutturare i dati, gestire in modo nativo strutture gerarchiche, elaborare e presentare i dati con strumenti xml nativi, validare i tipi di strutture e i tipi di dati consentiti, gestire riferimenti incrociati tramite opportuni meccaniscmi di dereferenziazione, aggiungere e gestire annotazioni a vari livelli di granularità.
	\end{block}


\end{frame}


% Da Slide Chiara
% XPath: XML Path Language
% XSL: eXtensible Stylesheet Language
% XSL-T: eXtensible Stylesheet Lang. – Transformations
% XSL-FO: eXtensible Stylesheet Lang. – Formatting Objects
% XQuery: XML Query language for XML Databases
% XInclude: XML inclusion language
% RelaxNG: Regular Expression Language for XML (New Generation)

% Da slide Vitali:
% XML allows to create markup languages that are
% readable, generic, structured, hierarchical.
% – Data: no problem
% – Hierarchical data: no problem
% – Text: no problem
% – Presentation of data: after transformation via XSLT
% – Hierarchical text: no problem
% – Validation: no problem
% – References: no problem
% – Annotations: as attributes or ad hoc sections of the
% document 