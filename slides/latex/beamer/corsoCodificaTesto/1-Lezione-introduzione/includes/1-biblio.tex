% bibliografia di riferimento
%% Ciotti
%% Burnard
%% TEI guide lines
%% Pierazzo (due libri)
%% Slide (del corso)
%% XML specification e technical report W3C (https://www.w3.org/TR/xml/)
%% XML visual
%% XSL XPATH
%% XSD (art of XSD - SQL validation)
%% DTD (libro visual XML)
%% RELAXNG (libro relaxng, tutorial)

%bibliografia
\begin{frame}
    \frametitle{References}
    \addtocounter{nframe}{1}
    \begin{thebibliography}{10}

        \setbeamertemplate{bibliography item}[paper]
        \tiny\bibitem{ciotti2012} Ciotti F., e Crupi G, a c. di. 2012. Dall’Informatica umanistica alle culture digitali. ROMA : Casa Editrice Università La Sapienza. \href{http://www.editricesapienza.it/node/7688}{open access: http://www.editricesapienza.it/node/7688}
        \tiny\bibitem{orlandi2010} Orlandi, T. (2010). Informatica testuale: teoria e prassi. Laterza.
        \tiny\bibitem{Pierazzo2015} Pierazzo, E. (2015). Digital Scholarly Editing : Theories, Models and Methods. Farnham Surrey: Ashgate.
        \tiny\bibitem{Pierazzo2016} Driscoll, M. J., and Pierazzo, E. (Eds.). (2016). Digital Scholarly Editing: Theories and Practices (Vol. 4). Open Book Publishers.
        \tiny\bibitem{Williams2009} Williams, I. (2009). Beginning XSLT and XPath: Transforming XML Documents and Data. Wiley.
        \tiny\bibitem{Kay2011} Kay, M. (2011). XSLT 2.0 and XPath 2.0 Programmer’s Reference. Wiley.

        \setbeamertemplate{bibliography item}[online]
        \tiny\bibitem{MSDN} \textit{XML Standards Reference}, MSDN. \url{https://msdn.microsoft.com/en-us/library/ms256177(v=vs.110).aspx}
        \tiny\bibitem{IBMXML1} IBM XML \textit{Tutorial}, \url{https://www.ibm.com/developerworks/xml/tutorials/xmlintro/xmlintro.html}

    \end{thebibliography}

\end{frame}