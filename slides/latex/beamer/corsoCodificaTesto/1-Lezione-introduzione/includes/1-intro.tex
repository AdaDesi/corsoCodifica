% slide di presentazione
%% chi sono, piccola bio, di cosa mi occupo 

% sezione intro frame 01
\begin{frame}
    \frametitle{Introduzione al Corso di Codifica dei Testi}
    \framesubtitle{Obiettivi, competenze e conoscenze}
    \addtocounter{nframe}{1}
    
    \begin{block}{Obiettivo}
        Illustrare i principi di modellazione e le prassi di codifica del testo per una adeguata rappresentazione ed elaborazione digitale di risorse testuali.  
    \end{block}

    \begin{block}{Rationale}
       Fornire gli strumenti e le conoscenze necessarie per progettare e realizzare criticamente una codifica digitale di testi complessi, in particolare testi letterari e di interesse storico-culturale, usando le linee guida della Text Encoding Initiative (TEI).
    \end{block}

\end{frame}

% sezione intro frame 02
\begin{frame}
    \frametitle{Argomenti trattati}
    \framesubtitle{Obiettivi, competenze e conoscenze}
    \addtocounter{nframe}{1}
    
    \begin{block}{Competenze attese}
        Al termine del corso sarete in grado di valutare il metodo di codifica più appropriato allo scenario d'interesse, di creare uno schema di codifica TEI e di usare gli strumenti più idonei per la codifica e la (semplice) elaborazione e visualizzazione di un testo.
    \end{block}

    

\end{frame}

% sezione intro frame 03
\begin{frame}
    \frametitle{Principali Argomenti}
    \framesubtitle{Obiettivi, competenze e conoscenze}
    \addtocounter{nframe}{1}

    
        \begin{itemize}
            \item Codifica dei caratteri e di testi
            \item I linguaggi di markup e introduzione a XML
            \item Creazione di schemi di codifica
            \item Le norme TEI (Text Encoding Initiative)
            \item Alcuni specifici Moduli TEI
            \item Definizione di schemi di codifica personalizzati
            \item introduzione ai fogli di stile XSLT
            \item elaborazione documenti XML-TEI (XSLT2.0, DOM)
            \item esempi, esercitazioni e seminari 
        \end{itemize}
    

\end{frame}


% sezione intro frame 04
\begin{frame}
    \frametitle{Perché è importante la codifica dei testi}
    \framesubtitle{Motivazioni}
    \addtocounter{nframe}{1}
    
    \begin{block}{Perché codificare}
    Il rapporto tra sapere umanistico e informatica non è solo una questione meramente strumentale. L'informatica non è solo un utensile dalle notevoli capacità.
    \\ Salto di paradigma sia dal punto di vista teorico e metodologico sia da quello pratico.
    \\ L’attività di codifica, dunque, assume una funzione metodologica nell’ambito delle discipline che si occupano del testo, e il linguaggio di codifica adottato può essere considerato come un linguaggio teorico.
    \\ esplicitare le sue ipotesi interpretative su un certo oggetto di studio
    \end{block}


\end{frame}

% sezione intro frame 05
\begin{frame}
    \frametitle{Perché è importante la codifica dei testi}
    \framesubtitle{Motivazioni}
    \addtocounter{nframe}{1}
    
    \begin{block}{Perché codificare}
    The problem: representing text digitally
• The solutions: we have a multitude of formats
• The discussion: find the best format
• The bad: there is no best format
• The good: all formats are, in a way, equivalent
• The deduction: there is a way in which formats
are not equivalent
• The odd: the non-equivalence is aesthetic rather
than substantive
• The take-home message: aesthetics is
important in representing text digitally 
    \end{block}

    

\end{frame}

% Nella nostra cultura la quasi totalità dei testi è veicolata da docu- menti materiali di varia natura e forma. Per rendere disponibile questo patrimonio attraverso i sistemi elettronici di gestione dell’informazio- ne è necessario dunque effettuare una trasposizione dei testi dal loro supporto originario al nuovo supporto elettronico.