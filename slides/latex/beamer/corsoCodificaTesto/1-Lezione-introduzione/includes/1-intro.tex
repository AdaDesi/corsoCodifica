
% sezione intro frame 01
\begin{frame}
    \frametitle{Introduzione al Corso di Codifica di Testi}
    \framesubtitle{Obiettivi, competenze e conoscenze}
    \addtocounter{nframe}{1}
    
    \begin{block}{Obiettivo}
        Illustrare i principi di modellazione e le prassi di codifica del testo per una adeguata rappresentazione ed elaborazione digitale di risorse testuali.  
    \end{block}

    \begin{block}{Rationale}
       Fornire gli strumenti e le conoscenze necessarie per progettare e realizzare criticamente una codifica digitale di testi complessi, in particolare testi letterari e di interesse storico-culturale, usando le linee guida della Text Encoding Initiative (TEI).
    \end{block}

\end{frame}

% sezione intro frame 02
\begin{frame}
    \frametitle{Argomenti trattati}
    \framesubtitle{Obiettivi, competenze e conoscenze}
    \addtocounter{nframe}{1}
    
    \begin{block}{Competenze attese}
        Al termine del corso sarete in grado di valutare il metodo di codifica più appropriato allo scenario d'interesse, di creare uno schema di codifica TEI e di usare gli strumenti più idonei per la codifica e la (semplice) elaborazione e visualizzazione di un testo.
    \end{block}

\end{frame}

% sezione intro frame 03
\begin{frame}
    \frametitle{Principali Argomenti}
    \framesubtitle{Obiettivi, competenze e conoscenze}
    \addtocounter{nframe}{1}

    
        \begin{itemize}
            \item Codifica dei caratteri e di testi
            \item I linguaggi di markup e introduzione a XML
            \item Creazione di schemi di codifica
            \item Le norme TEI (Text Encoding Initiative)
            \item Alcuni specifici Moduli TEI
            \item Definizione di schemi di codifica personalizzati
            \item introduzione ai fogli di stile XSLT
            \item elaborazione documenti XML-TEI (XSLT2.0, DOM)
            \item esempi, esercitazioni e seminari 
        \end{itemize}

\end{frame}


% sezione intro frame 04
\begin{frame}
    \frametitle{Perché è importante la codifica dei testi}
    \framesubtitle{Motivazioni teoriche}
    \addtocounter{nframe}{1}
    
    \begin{block}{Perché codificare}
        Il rapporto tra sapere umanistico e informatica non è solo una questione meramente strumentale. 
        \\ L'informatica non è solo un utensile dalle notevoli capacità.
        \\ Salto di paradigma sia dal punto di vista teorico e metodologico sia da quello pratico.

        L'attività di codifica come funzione metodologica nell'ambito delle discipline che si occupano del testo.
        \\ Il linguaggio di codifica adottato può essere considerato come un linguaggio teorico.
        \\ Esplicitare le ipotesi interpretative su un certo oggetto di studio
    \end{block}

\end{frame}

% sezione intro frame 05
\begin{frame}
    \frametitle{Perché è importante la codifica dei testi}
    \framesubtitle{Motivazioni pratiche}
    \addtocounter{nframe}{1}
    
    \begin{block}{Perché codificare}
        Nella nostra cultura la quasi totalità dei testi è \underline{registrata} su materiali fisici di varia natura e forma (manoscritti su pietra, pergamena, papiri, carta, libri a stampa, incunabula, cinquecentine, etc).

        Per rendere disponibile questo patrimonio attraverso i sistemi per la gestione dell'informazione digitali e computazionali è necessario effettuare una trasposizione/trascodifica dei testi (procedimento di conversione dei dati codificati secondo un sistema verso un sistema diverso) dal loro supporto originario verso il nuovo supporto elettronico.
    \end{block}

\end{frame}

% sezione intro frame 06
\begin{frame}
    \frametitle{Perché è importante la codifica dei testi}
    \framesubtitle{In sintesi}
    \addtocounter{nframe}{1}
    
    \begin{block}{Perché codificare}
    Il focus del corso sarà diretto alla rappresentazione digitale del testo.
    \\ Per ottenere tale rappresentazione ci sono diversi formati e formalismi:
    \\ la nostra scelta ricade sulle norme suggerite dal consorzio TEI.
    \\ molte questioni ancora non risolte e controverse, sia teorico-metodologico, sia pratico-tecnologico.
    \\ I formalismi e le tecnologie adottate possono essere viste come isomorfe.
    \end{block}

\end{frame}

% sezione intro frame 07
\begin{frame}
    \frametitle{Perché è importante la codifica dei testi}
    \framesubtitle{Ma in definitiva}
    \addtocounter{nframe}{1}
    
    \begin{block}{Perché codificare}

        \begin{center}
            \textit{Le differenze di formato sono più che altro estetiche e non sostanziali}
        \end{center}

    \end{block}
     

    \begin{block}{Perché codificare}

        \begin{center}
            \textbf{Ma anche l'occhio \underline{umano} vuole la sua parte}
        \end{center}
       

    \end{block}

\end{frame}

