\begin{frame}
	\frametitle{Intro Text Encoding Initiative}
	\framesubtitle{TEI}
	\addtocounter{nframe}{1}

	\begin{block}{Nota sugli elementi illustrati}
    \begin{itemize}
        \item punto di riferimento per il corso: la TEI P5 “completa”
        \item iniziamo con la TEI Lite (= versione “leggera”)
        \item proposti elementi di uso generale e di uso specifico
        \item per cominciare, la TEI Lite è più semplice da gestire
        \item elementi non presenti di default evidenziati 
    \end{itemize}
       
       
        
       
        
       
    \end{block}
\end{frame}

\begin{frame}
	\frametitle{Intro Text Encoding Initiative}
	\framesubtitle{TEI}
	\addtocounter{nframe}{1}

	\begin{block}{Nota sugli elementi illustrati}
        \begin{itemize}
            \item in caso di dubbio consultare le liste degli elementi:
            \item TEI Lite: \url{http://www.tei-c.org/release/doc/tei-p5-exemplars/html/teilite.doc.html}
            \item TEI P5:\url{ http://www.tei-c.org/release/doc/tei-p5-doc/en/html/REF-ELEMENTS.html}
        \end{itemize}
        
        \textit{\textbf{NOTA}: quando si crea lo schema controllare che gli elementi
        siano effettivamente inclusi nei moduli prescelti}
    \end{block}
\end{frame}


\begin{frame}
	\frametitle{Intro Text Encoding Initiative}
	\framesubtitle{Schemi di codifica TEI – Moduli base}
	\addtocounter{nframe}{1}

	\begin{block}{TEI}
        
        Suggerimenti riguardo gli esercizi
        non fate copia e incolla da Wordpad o Notepad, aprite
        direttamente i testi usando l'editor
        i documenti XML hanno come estensione .xml, quindi
        salvate subito una copia con tale estensione
        aprite le Guidelines TEI nel browser
        per prima cosa marcare la struttura: cominciare la
        marcatura semantica come secondo passo
        ogni esercizio ha degli obiettivi da raggiungere:
        controllate prima di spedire
        non lasciate accumulare gli errori
    \end{block}
    
   

\end{frame}