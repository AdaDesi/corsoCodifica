
% esempio TEI descrizione file
% tre componenti obbligatorie: title statement, publication statement, descrizione della fonte.

% <titleStmt>
%    <title xml:lang="sk">Yogadarśanam (arthāt yogasūtrapūphah).</title>
%    <title>The Yoga sūtras of Patañjali: a digital edition.</title>
%    <author>Patañjali</author>
%    <funder>Wellcome Institute for the History of Medicine</funder>
%    <principal>Dominik Wujastyk</principal>
%    <respStmt>
%        <name>Wieslaw Mical</name>
%        <resp>data entry and proof</resp>
%    </respStmt>
%    <respStmt> 
%        <name>Jan Hajic</name> 
%        <resp>conversion to TEI-conformant markup</resp> 
%    </respStmt>
% </titleStmt>

% esempio publication statement

%% Descrizione della fonte
% document formally the object or objects from which the TEI document has been derived, using traditional bibliographic
% Born Digital
% Printed Source
% sbobbinatura
% manoscritto (msDesc)

%% Esempio di sourceDesc
% <sourceDesc>
%    <bibl xml:id="Sue1846">
%        <author>
%            <surname>Sue</surname>,
%            <forename>Eugène</forename>
%        </author>
%        <title level="m">Martin, l’enfant trouvé : Mémoires d’un valet de chambre</title>
%        <imprint>
%            <publisher>C. Muquardt</publisher>
%            <pubPlace>Bruxelles</pubPlace>
%            <pubPlace>Leipzig</pubPlace>
%            <date when="1846">MDCCCXLVI</date>
%        </imprint>
%    </bibl>
% </sourceDesc>