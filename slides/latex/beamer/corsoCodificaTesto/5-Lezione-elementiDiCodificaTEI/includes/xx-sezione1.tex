% xx sezione 1 frame 01
\begin{frame}
    \frametitle{Attributi Globali}
    \framesubtitle{Elenco}
    \addtocounter{nframe}{1}


\textbf{\textrm{att.global} provides attributes common to all elements in the TEI encoding scheme.}

\begin{description}
    \item [@xml:id]     \textbf{identifier} provides a unique identifier for the element bearing the attribute.
    \item [@n]          \textbf{number} gives a number (or other label) for an element, which is not necessarily unique within the document.
    \item [@xml:lang]   \textbf{language} indicates the language of the element content using a ‘tag’ generated according to BCP 47\footnote{see \href{http://google.con}{http://google.com}}.
\end{description}

\end{frame}


% xx sezione 1 frame 02
\begin{frame}
    \frametitle{Attributi Globali}
    \framesubtitle{Elenco cont..}
    \addtocounter{nframe}{1}


\textbf{\textmd{att.global} provides attributes common to all elements in the TEI encoding scheme.}

\begin{description}
    \item rend [att.global.rendition]	(rendition) indicates how the element in question was rendered or presented in the source text.
    \item style [att.global.rendition]	contains an expression in some formal style definition language which defines the rendering or presentation used for this element in the source text
    \item rendition [att.global.rendition]	points to a description of the rendering or presentation used for this element in the source text.
\end{description}

\end{frame}

% xx sezione 1 frame 03
\begin{frame}
    \frametitle{Attributi Globali}
    \framesubtitle{Elenco cont...}
    \addtocounter{nframe}{1}


\textbf{\textmd{att.global} provides attributes common to all elements in the TEI encoding scheme.}

\begin{description}
    \item xml:base	provides a base URI reference with which applications can resolve relative URI references into absolute URI references.
    \item xml:space	signals an intention about how white space should be managed by applications.
    \item source [att.global.source]	specifies the source from which some aspect of this element is drawn.
\end{description}

\end{frame}

% xx sezione 1 frame 04
\begin{frame}
    \frametitle{Attributi Globali}
    \framesubtitle{Elenco cont....}
    \addtocounter{nframe}{1}


\textbf{\textmd{att.global} provides attributes common to all elements in the TEI encoding scheme.}

\begin{description}
    \item  cert [att.global.responsibility]	(certainty) signifies the degree of certainty associated with the intervention or interpretation.
    \item resp [att.global.responsibility]	(responsible party) indicates the agency responsible for the intervention or interpretation, for example an editor or transcriber.
\end{description}

\end{frame}


% xx sezione 1 frame 05

\begin{frame} [fragile]
    \frametitle{Attributi Globali}
    \framesubtitle{Esempio \textrm{@xml:lang}}
    \addtocounter{nframe}{1}

    \textbf{\textrm{xml:lang} indica la lingua e il sistema di scrittura usato}
    \defverbatim{\langatt}{%
        \begin{tiny}
        \begin{verbatim}
            <TEI xmlns="http://www.tei-c.org/ns/1.0">
                <teiHeader xml:lang="en">
                    <!-- ... -->
                </teiHeader>
                <text xml:lang="fr">
                    <body>
                        <div>
                            <!-- chapter one is in French -->
                        </div>
                        <div xml:lang="de">
                            <!-- chapter two is in German -->
                        </div>
                        <div>
                            <!-- chapter three is French -->
                        </div>
                        <!-- ... -->
                    </body>
                </text>
            </TEI>
        \end{verbatim}
        \end{tiny}
        }
        \begin{block}{XML TEI con esempio di uso \textrm{xml:lang} attribute}
            {\langatt}
        \end{block}
\end{frame}


% xx sezione 1 frame 06
\begin{frame}
    \frametitle{Attributi Globali}
    \framesubtitle{Stile e Aspetto}
    \addtocounter{nframe}{1}

    
    \textbf{In the TEI scheme, it is possible to supply information about the appearance of elements within a source document in the following distinct ways:}

    \begin{itemize}
        \item One or more properties may be specified as the default for a set of elements (based on an external scheme, by default CSS), using rendition elements and their selector attributes;
        \item One or more properties may be specified for individual element occurrences, using the rend attribute with any convenient set of one or more sequence-indeterminate tokens;
    \end{itemize}
\end{frame}

% xx sezione 1 frame 07
\begin{frame}
    \frametitle{Attributi Globali}
    \framesubtitle{Stile e Aspetto cont..}
    \addtocounter{nframe}{1}
    
    \textbf{Note that these TEI attributes always describe the rendition or appearance of the source document, not intended output renditions, although often the two may be closely related.}

    \begin{itemize}
        \item One or more properties may be specified for individual element occurrences, using the rendition attribute to point to rendition elements;
        \item One or more properties may be supplied explicitly for individual element occurrences, using the style attribute.
    \end{itemize}

\end{frame}


% xx sezione 1 frame 08

\begin{frame}
    \frametitle{Attributi Globali}
    \framesubtitle{da vari altri Moduli Tabella}
    \addtocounter{nframe}{1}
    %fare una tabella 
class name	module name	see further
att.global.linking	linking	16 Linking, Segmentation, and Alignment
att.global.analytic	analysis	17 Simple Analytic Mechanisms
att.global.facs	transcr	11.1 Digital Facsimiles
att.global.change	transcr	11.7 Identifying Changes and Revisions

\end{frame}
