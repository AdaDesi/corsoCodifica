%% frame 01
% Every TEI document must have a TEI Header, represented by a <teiHeader> element.
% The TEI header has four main components,
% file description, encoding description, profile description, revision description

% esempio TEI header Minimale
% <teiHeader>
%    <fileDesc>
%        <titleStmt>
%            <title>Title of the work</title>
%        </titleStmt>
%        <publicationStmt>
%            <p>Information about the publication of the work</p>
%        </publicationStmt>
%        <sourceDesc>
%            <p>Information about the source from which the work was derived</p>
%        </sourceDesc>
%    </fileDesc>
% </teiHeader>

% Rifarsi al capitolo due delle linee guida per una descrizione esaustiva e sistematica

\begin{frame}
	\frametitle{Intro Text Encoding Initiative}
	\framesubtitle{Schemi di codifica TEI – Intestazione}
	\addtocounter{nframe}{1}

    \begin{block}{Contenuto del TEI header}
        \textbf{Qualsiasi documento TEI deve includere una intestazione (TEIHeader)}.
    \end{block}

    \begin{block}{TEI header: componenti}
        file description, encoding description, profile description, revision description, no TEI metadata
    \end{block}

\end{frame}

\begin{frame}
	\frametitle{Intro Text Encoding Initiative}
	\framesubtitle{Schemi di codifica TEI – Intestazione}
	\addtocounter{nframe}{1}

    \begin{block}{Contenuto del TEI header}
        \begin{itemize}
            \item metadati relativi al documento (utili per collezioni di testi
            codificati)
            \item descrizione del file usando \texttt{<fileDesc>} (obbligatoria)
            \item descrizioni relative al tipo di codifica, al contenuto del
            documento, alle sue revisioni (facoltative)
        \end{itemize}
    \end{block}
\textit{E' possibile includere testi introduttivi e spiegazioni relative alla
codifica effettuata (preziosi per l’interscambio!)}

\end{frame}





\begin{frame}
	\frametitle{Intro Text Encoding Initiative}
	\framesubtitle{Schemi di codifica TEI – Contentuto}
	\addtocounter{nframe}{1}

    \begin{block}{Testo in prosa}
        Most elements contain simple running prose at some level. Many elements may contain either prose (possibly organized into paragraphs) or more specific elements, which themselves contain prose. In this chapter's descriptions of element content, the phrase prose description should be understood to imply a series of paragraphs, each marked as a p element. The word phrase, by contrast, should be understood to imply character data, interspersed as need be with phrase-level elements, but not organized into paragraphs.
    \end{block}
\textit{}

\end{frame}


\begin{frame}
	\frametitle{Intro Text Encoding Initiative}
	\framesubtitle{Schemi di codifica TEI – Contenuto}
	\addtocounter{nframe}{1}

    \begin{block}{Grouping elements}
        Elements whose names end with the suffix Stmt (e.g. editionStmt, titleStmt) and the xenoData element enclose a group of specialized elements recording some structured information. In the case of the bibliographic elements, the suffix Stmt is used in names of elements corresponding to the ‘areas’ of the International Standard Bibliographic Description.5 In the case of the xenoData element, the specialized elements are not TEI elements, but rather come from some other metadata scheme. In most cases grouping elements may contain prose descriptions as an alternative to the set of specialized elements, thus allowing the encoder to choose whether or not the information concerned should be presented in a structured form or in prose.
    \end{block}
\textit{}

\end{frame}


\begin{frame}
	\frametitle{Intro Text Encoding Initiative}
	\framesubtitle{Schemi di codifica TEI – Contenuto}
	\addtocounter{nframe}{1}

    \begin{block}{Dichiarazioni}
        Elements whose names end with the suffix Decl (e.g. tagsDecl, refsDecl) enclose information about specific encoding practices applied in the electronic text; often these practices are described in coded form. Typically, such information takes the form of a series of declarations, identifying a code with some more complex structure or description. A declaration which applies to more than one text or division of a text need not be repeated in the header of each such text or subdivision. Instead, the decls attribute of each text (or subdivision of the text) to which the declaration applies may be used to supply a cross-reference to it, as further described in section 15.3 Associating Contextual Information with a Text.
    \end{block}
\textit{}

\end{frame}


\begin{frame}
	\frametitle{Intro Text Encoding Initiative}
	\framesubtitle{Schemi di codifica TEI – Contenuto}
	\addtocounter{nframe}{1}

    \begin{block}{Descrizioni}

        Elements whose names end with the suffix Desc (e.g. settingDesc, projectDesc) contain a prose description, possibly, but not necessarily, organized under some specific headings by suggested sub-elements.

    \end{block}
\textit{}

\end{frame}



\begin{frame}
	\frametitle{Intro Text Encoding Initiative}
	\framesubtitle{Schemi di codifica TEI – Moduli base}
	\addtocounter{nframe}{1}

	\begin{block}{documento TEI - schema di intestazione TEI minima}
        \texttt{<teiHeader>}
        \texttt{ <fileDesc>}
        \texttt{  <titleStmt>...</titleStmt>}
        \texttt{  <publicationStmt>...</publicationStmt>}
        \texttt{  <sourceDesc>...</sourceDesc>}
        \texttt{ </fileDesc>}
        \texttt{</teiHeader>}
    \end{block}
    

\end{frame}


\begin{frame}
	\frametitle{Intro Text Encoding Initiative}
	\framesubtitle{Schemi di codifica TEI – Intestazione}
	\addtocounter{nframe}{1}

	\begin{block}{documento TEI - schema di intestazione TEI minima}
        Metadati essenziali riguardo il titolo, la modalità di diffusione e
        la fonte originaria di un testo codificato.
        \\Permettono classificazione, archiviazione ed elaborazione
        bibliografica
    \end{block}
    
\end{frame}



\begin{frame}
	\frametitle{Intro Text Encoding Initiative}
	\framesubtitle{Schemi di codifica TEI – Intestazione}
	\addtocounter{nframe}{1}

        \texttt{<teiHeader>
        <fileDesc>
        <titleStmt>
        <title>La Divina Commedia: versione elettronica</title>
        <respStmt>
        <resp>Conversione TEI P5 a cura di</resp><name>M. Rossi</name>
        </respStmt>
        </titleStmt>
        <publicationStmt>
        <publisher>Università di Pisa</publisher>
        <date>2002-11-07</date>
        <availability status=``restricted''><p></p></availability>
        </publicationStmt>
        <sourceDesc>
        <bibl><title>La Divina Commedia</title><author>Dante Alighieri
        </author><publisher>Mondadori</publisher>
        <date>1988</date></bibl>
        </sourceDesc>
        </fileDesc>
        </teiHeader>}

\end{frame}



\begin{frame}
	\frametitle{Intro Text Encoding Initiative}
	\framesubtitle{Schemi di codifica TEI – Intestazione}
	\addtocounter{nframe}{1}

    \begin{block}{Le altre componenti dell’intestazione TEI}
        \begin{itemize}
            \item \texttt{<encodingDesc>} informazioni riguardo lo schema (e il
            modello di codifica) utilizzato
            \item  \texttt{<profileDesc>} descrizione del testo: quando è stato
            creato, da chi, usando quali lingue etc.
            \item \texttt{<revisionDesc>} informazioni sulle versioni del file
        \end{itemize}
    \end{block}
    \textit{I metadati sono una componente essenziale di qualsiasi
        progetto di digitalizzazione}
\end{frame}


\begin{frame}
	\frametitle{Intro Text Encoding Initiative}
	\framesubtitle{Schemi di codifica TEI – Moduli base}
	\addtocounter{nframe}{1}

	\begin{block}{Errori frequenti}
        \textit{Si fraintende il significato dell’elemento \texttt{<fileDesc>}}
        \begin{itemize}
            \item serve in primo luogo a dare informazioni sul file stesso, non sul testo originale
            \item il riferimento alla fonte dalla quale è tratto il testo codificato
            deve essere inserito nel \texttt{<sourceDesc>}
        \end{itemize}
    \end{block}

\end{frame}

\begin{frame}
	\frametitle{Intro Text Encoding Initiative}
	\framesubtitle{Schemi di codifica TEI – Intestazione}
	\addtocounter{nframe}{1}

        \textbf{Esempio Intestazione TEI raccomandata}
        \textit{Tratto dalle linee Guida della TEI. Capitolo 2 paragrafo 2.7}
        \\\url{}

\end{frame}

% file description 
%\subsection{File Description}
%
 esempio TEI descrizione file
% tre componenti obbligatorie: title statement, publication statement, descrizione della fonte.

% <titleStmt>
%    <title xml:lang="sk">Yogadarśanam (arthāt yogasūtrapūphah).</title>
%    <title>The Yoga sūtras of Patañjali: a digital edition.</title>
%    <author>Patañjali</author>
%    <funder>Wellcome Institute for the History of Medicine</funder>
%    <principal>Dominik Wujastyk</principal>
%    <respStmt>
%        <name>Wieslaw Mical</name>
%        <resp>data entry and proof</resp>
%    </respStmt>
%    <respStmt> 
%        <name>Jan Hajic</name> 
%        <resp>conversion to TEI-conformant markup</resp> 
%    </respStmt>
% </titleStmt>

% esempio publication statement

%% Descrizione della fonte
% document formally the object or objects from which the TEI document has been derived, using traditional bibliographic
% Born Digital
% Printed Source
% sbobbinatura
% manoscritto (msDesc)

%% Esempio di sourceDesc
% <sourceDesc>
%    <bibl xml:id="Sue1846">
%        <author>
%            <surname>Sue</surname>,
%            <forename>Eugène</forename>
%        </author>
%        <title level="m">Martin, l’enfant trouvé : Mémoires d’un valet de chambre</title>
%        <imprint>
%            <publisher>C. Muquardt</publisher>
%            <pubPlace>Bruxelles</pubPlace>
%            <pubPlace>Leipzig</pubPlace>
%            <date when="1846">MDCCCXLVI</date>
%        </imprint>
%    </bibl>
% </sourceDesc>

% encoding description
%\subsection{Encoding Description}
%% Encoding descsription (<encodingDesc />) is used to supply information about almost any aspect of the encoding process itself

%% esempio descrizione della codifica
% <encodingDesc>
%    <projectDesc>
%        <p>Texts collected for use in the Claremont Shakespeare Clinic, June 1990.</p>
%    </projectDesc>
%    <samplingDecl>
%        <p>Each text contains a sample of up to 2000 words, running from the start of the document to the end of the sentence
%            after the 2000 word mark. For the purposes of word counting, hyphens and apostrophes were treated as spaces.
%            </p>
%    </samplingDecl>
%    <editorialDecl>
%        <normalization>
%            <p>Word forms broken by end of line hyphenation have been reconstructed without comment. The hyphen has been removed
%                except for hyphenated forms attested elsewhere in the text. </p>
%        </normalization>
%        <quotation marks="all" form="std">
%            <p>All quotation marks have been removed. Direct speech is represented by the use of the
%                <gi>said</gi> tag; other quoted material is represented by means of the
%                <gi>q</gi> tag. </p>
%        </quotation>
%    </editorialDecl>
% </encodingDesc>

%% for human reader and for automated process

% esempio charDecl
% <charDecl>
%    <glyph xml:id="z103">
%        <glyphName>LATIN LETTER Z WITH TWO STROKES</glyphName>
%        <mapping type="standardized">z</mapping>
%        <mapping type="PUA">U+E304</mapping>
%    </glyph>
% </charDecl>

% segue nel testo 
% <p> ... mulct<g ref="#z103"/> ... </p>

% tassonomia

% <classDecl>
%    <taxonomy xml:id="size">
%        <category xml:id="large">
%            <catDesc>story occupies more than half a page</catDesc>
%        </category>
%        <category xml:id="medium">
%            <catDesc>story occupies between quarter and a half page</catDesc>
%        </category>
%        <category xml:id="small">
%            <catDesc>story occupies less than a quarter page</catDesc>
%        </category>
%        <!-- etc -->
%    </taxonomy>
%    <taxonomy xml:id="topic">
%        <category xml:id="politics-domestic">
%            <catDesc>Refers to domestic political events</catDesc>
%        </category>
%        <category xml:id="politics-foreign">
%            <catDesc>Refers to foreign political events</catDesc>
%        </category>
%        <category xml:id="social-women">
%            <catDesc>refers to role of women in society</catDesc>
%        </category>
%        <category xml:id="social-servants">
%            <catDesc>refers to role of servants in society</catDesc>
%        </category>
%        <!-- etc -->
%    </taxonomy>
% </classDecl>

% uso nel testo 
% <catRef target="#small #social-women"/>

%% esempio dichiarazione dei tag
% <tagsDecl>
%    <rendition xml:id="IT" scheme="css">font-style:italic</rendition>
%    <rendition xml:id="FontRoman" scheme="css">font-family: serif</rendition>
%    <namespace name="http://www.tei-c.org/ns/1.0">
%        <tagUsage gi="emph" render="#IT" />
%        <tagUsage gi="hi" render="#IT" />
%        <tagUsage gi="text" render="#FontRoman" />
%    </namespace>
% </tagsDecl>

% uso nel testo
% @rendition attribute

% profile description
%\subsection{Profile Description}
%% elemento che descrive il profilo del documento TEI
% elementi di metadatazione non bibliografici
% elementi per edentificare i vari passaggi genetici elemento <listChange />
 %% esempio proust di Pierazzo
 %% esempio carteggio

 %% esempio profile
% <profileDesc>
%    <creation>
%        <date when="1962" /> </creation>
%    <textClass>
%        <catRef target="#WRI #ALLTIM1 #ALLAVA2 #ALLTYP3 #WRIDOM5 #WRILEV2 #WRIMED1 #WRIPP5 #WRISAM3 #WRISTA2 #WRITAS0" />
%        <classCode scheme="DLEE">W nonAc: humanities arts</classCode>
%        <keywords scheme="COPAC">
%            <term>History, Modern - 19th century</term>
%            <term>Capitalism - History - 19th century</term>
%            <term>World, 1848-1875</term>
%        </keywords>
%    </textClass>
% </profileDesc>

% revision description
%\subsection{Revision Description}
%% represented by a <revisionDesc> element which contains a list of <change>
% given first. The <listChange> element mentioned above may also be used here to refer to identified stages in the evolution of the electronic file

% esempio
% <revisionDesc>
%    <listChange>
%        <change when="2013-05-11">First complete draft</change>
%        <change when="2013-04-07">Created header and document structure</change>
%    </listChange>
% </revisionDesc>

% no tei metadata
%\subsection{No TEI Metadata}
%\input{includes/subinclude/noTEIData.tex}
