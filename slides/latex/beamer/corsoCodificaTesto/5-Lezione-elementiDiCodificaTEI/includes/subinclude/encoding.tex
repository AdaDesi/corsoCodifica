% Encoding descsription (<encodingDesc />) is used to supply information about almost any aspect of the encoding process itself

%% esempio descrizione della codifica
% <encodingDesc>
%    <projectDesc>
%        <p>Texts collected for use in the Claremont Shakespeare Clinic, June 1990.</p>
%    </projectDesc>
%    <samplingDecl>
%        <p>Each text contains a sample of up to 2000 words, running from the start of the document to the end of the sentence
%            after the 2000 word mark. For the purposes of word counting, hyphens and apostrophes were treated as spaces.
%            </p>
%    </samplingDecl>
%    <editorialDecl>
%        <normalization>
%            <p>Word forms broken by end of line hyphenation have been reconstructed without comment. The hyphen has been removed
%                except for hyphenated forms attested elsewhere in the text. </p>
%        </normalization>
%        <quotation marks="all" form="std">
%            <p>All quotation marks have been removed. Direct speech is represented by the use of the
%                <gi>said</gi> tag; other quoted material is represented by means of the
%                <gi>q</gi> tag. </p>
%        </quotation>
%    </editorialDecl>
% </encodingDesc>

%% for human reader and for automated process

% esempio charDecl
% <charDecl>
%    <glyph xml:id="z103">
%        <glyphName>LATIN LETTER Z WITH TWO STROKES</glyphName>
%        <mapping type="standardized">z</mapping>
%        <mapping type="PUA">U+E304</mapping>
%    </glyph>
% </charDecl>

% segue nel testo 
% <p> ... mulct<g ref="#z103"/> ... </p>

% tassonomia

% <classDecl>
%    <taxonomy xml:id="size">
%        <category xml:id="large">
%            <catDesc>story occupies more than half a page</catDesc>
%        </category>
%        <category xml:id="medium">
%            <catDesc>story occupies between quarter and a half page</catDesc>
%        </category>
%        <category xml:id="small">
%            <catDesc>story occupies less than a quarter page</catDesc>
%        </category>
%        <!-- etc -->
%    </taxonomy>
%    <taxonomy xml:id="topic">
%        <category xml:id="politics-domestic">
%            <catDesc>Refers to domestic political events</catDesc>
%        </category>
%        <category xml:id="politics-foreign">
%            <catDesc>Refers to foreign political events</catDesc>
%        </category>
%        <category xml:id="social-women">
%            <catDesc>refers to role of women in society</catDesc>
%        </category>
%        <category xml:id="social-servants">
%            <catDesc>refers to role of servants in society</catDesc>
%        </category>
%        <!-- etc -->
%    </taxonomy>
% </classDecl>

% uso nel testo 
% <catRef target="#small #social-women"/>

%% esempio dichiarazione dei tag
% <tagsDecl>
%    <rendition xml:id="IT" scheme="css">font-style:italic</rendition>
%    <rendition xml:id="FontRoman" scheme="css">font-family: serif</rendition>
%    <namespace name="http://www.tei-c.org/ns/1.0">
%        <tagUsage gi="emph" render="#IT" />
%        <tagUsage gi="hi" render="#IT" />
%        <tagUsage gi="text" render="#FontRoman" />
%    </namespace>
% </tagsDecl>

% uso nel testo
% @rendition attribute