% Portabilità e riutilizzabilità
% schema di codifica

%  I supporti testuali, dunque, debbono garantire e fa- cilitare questa universalità dell’accesso al loro contenuto.

% L’archiviazione su supporto digitale del patrimonio letterario e culturale delle culture mondiali deve misurarsi con questi problemi, e adottare degli schemi di codifica capaci di garantire la massima portabilità. 

% Una risorsa informativa digitale è portabile se è intercambiabile tra sistemi diversi, riutilizzabile in molteplici processi computazionali an- che a distanza di tempo, e integrabile da ulteriori risorse informative omogenee

%codifica dell’informazione gode delle seguenti proprietà:
%• indipendenza dall’hardware, ovvero da una particolare architettura elaborativa (processore), da un particolare supporto digitale (disco magnetico, disco ottico, etc.), o da un particolare dispositivo o sistema di output (video, stampa);
%• indipendenza dal software, sia sistemi operativi, sia applicazioni deputate alla creazione, analisi, manipolazione e visualizzazione di testi elettronici; (formati di dati proprietari mutamente incompatibili)
%• indipendenza logica dalle applicazioni ovvero indipendenza semantica dello schema di codifica da un particolare processo applicativo.

%  Esso deve divenire uno standard
% I vantaggi di uno standard formale o informale, oltre alla portabilità sta anche nella sua apertura, ovvero nella disponibilità pubblica delle sue specifiche.

% sistemi di word processing: wysiwyg
%%  questi sistemi sono nella maggior parte dei casi applicazioni di elaborazione testi, e dunque un documento prodot- to con questi sistemi non si presta facilmente a essere sottoposto a procedimenti computazionali di analisi testuale o di reperimento dell’informazione.

% La disposizione alla rappresentazione di strutture astratte non pone limiti alla natura e tipologia delle caratteristiche testuali che si posso- no codificare in un testo elettronico. Queste possono essere utilizzate indifferentemente 

% Se il linguaggio è dotato di una sintassi che permette di specificare le relazioni tra gli elementi, essa può essere usata per rappresentare la struttura e l’organizzazione del testo a un determinato livello di descrizione, o i rapporti tra ele- menti appartenenti a diversi livelli.

% offrono notevoli vantaggi nel trattamento auto- matico dei testi

% un sistema di codifica dichiarativa assista un autore nel processo di scrittura, poiché focalizza l’attenzione sul contenuto di un testo (o sulla struttura del contenuto) piuttosto che sulla sua forma grafica.

% I sistemi di markup dichiarativo introducono consistenti vantaggi anche nei processi produttivi editoriali e nella gestione dei flussi in- formativi aziendali. Poiché un medesimo schema di codifica dichiara- tivo può essere utilizzato in molteplici forme di trattamento informati- co, i costi di produzione e gestione di una base dati testuale vengono fortemente ridotti.

% I sistemi di codifica dichiarativa peraltro si prestano ottimamente per rappresentare strutture complesse come riferimenti incrociati e collegamenti tra elementi all’interno di un testo e tra più testi

% Infatti un database offre dei note- voli vantaggi dal punto di vista delle prestazioni computazionali e del- la velocità di ricerca, anche se richiede in generale una ingente quanti- tà di memoria per l’archiviazione.

% La codifica permette allo studioso di esplicitare le sue ipotesi inter- pretative.

% OHCO: efficienza computazionale che la struttura gerarchica mostra; 

% I linguaggi di markup dichiarativi permettono di predicare l’appartenenza di un dato segmento testuale a una classe di strutture testuali definita dall’utente;
%  In tal modo è pos- sibile descrivere formalmente le caratteristiche di un testo in modo in- dipendente da particolari finalità di trattamento
% da contingenti for- me di presentazione grafica su un qualsivoglia supporto fisico

% I linguaggi di markup dichiarativi, e in particolare SGML e XML, si sono rivelati dei veri e propri strumenti di supporto all’analisi com- putazionale dei testi
% la sintassi del linguaggio di codifica può essere usata per rappresentare le relazioni tra gli elementi strutturali di un testo, a un determinato livello di descrizione.