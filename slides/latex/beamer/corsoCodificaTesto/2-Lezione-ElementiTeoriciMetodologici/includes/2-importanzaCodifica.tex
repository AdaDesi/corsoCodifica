% Vedere Slide CHiara. Roberto
% Portabilità e riutilizzabilità
% schema di codifica
% TEI XML focuses on the meaning of text, rather than its appearance.

\begin{frame}
	\frametitle{Importanza della codifica digitale}
	\framesubtitle{Perché effettuare la codifica}
	\addtocounter{nframe}{1}

	\begin{block}{Digitalizzare un testo}
		Digitalizzare per poter favorire l'elaborazione e il trattamento automatico dei testi
	\end{block}

	\begin{block}{Trattamento dei testi}
        \begin{itemize}
            \item  analisi di tipo linguistico (linguistica computazionale,
            database testuali, corpora linguistics)
            \item analisi di altro tipo (metrica, stilistica, etc.)
            \item ricerca testuale avanzata
            \item pubblicazione in vari formati (sul web, come ebook, a
            stampa)
            \item didattica
        \end{itemize}
 
    \end{block}
\end{frame}

\begin{frame}
	\frametitle{Importanza della codifica digitale}
	\framesubtitle{Perché effettuare la codifica}
	\addtocounter{nframe}{1}

	\begin{block}{Digitalizzare un testo}
		Per facilitare e garantire una universalità di accesso al loro contenuto 
	\end{block}

	\begin{block}{Vantaggi della digitalizzazione}
        \begin{itemize}
            \item Edizioni elettroniche garantiscono diffusione capillare
            (via web) e nuove funzionalità (ipertesti, ricerca, etc.)
            \item Permettono anche di preservare i documenti più antichi
            (e fragili) riducendone la consultazione diretta
        \end{itemize}
     \end{block}
\end{frame}

\begin{frame}
	\frametitle{Importanza della codifica digitale}
	\framesubtitle{Perché effettuare la codifica}
	\addtocounter{nframe}{1}

	\begin{block}{Superare i problemi dei documenti digitali}
		\begin{itemize}
            \item disponibilità di hardware e software
            \item sistemi proprietari chiusi
            \item elevata obsolescenza e limitata manutenibilità
            \item difficile portabilità su piattaforme diverse
        \end{itemize}
	\end{block}
	
\end{frame}

\begin{frame}
	\frametitle{Importanza della codifica digitale}
	\framesubtitle{Perché effettuare la codifica}
	\addtocounter{nframe}{1}

	\begin{block}{Sistemi non adatti}
		\begin{itemize}
            \item word processing: WYSIWYG (What You See Is What You Get)
            \item sistemi proprietari chiusi (Word, Adobe, etc)
            \item elevata obsolescenza e limitata manutenibilità
            \item difficile portabilità su piattaforme diverse (windows, linux)
        \end{itemize}
	\end{block}
	
\end{frame}

\begin{frame}
	\frametitle{Importanza della codifica digitale}
	\framesubtitle{Perché effettuare la codifica}
	\addtocounter{nframe}{1}

	\begin{block}{Massimizzare le seguenti proprietà: Portabilità}
		\begin{itemize}
            \item indipendenza dall’hardware: processore, supporto, output
            \item indipendenza dal software: sistemi operativi, applicazioni di authoring, applicazioni di visualizzazione
            \item indipendenza dai sistemi di codifica dei caratteri
            \item indipendenza logica: da un particolare processo applicativo
        \end{itemize}
	\end{block}
	
\end{frame}




%La codifica dell’informazione gode delle seguenti proprietà:
% indipendenza dall’hardware, ovvero da una particolare architettura elaborativa (processore), da un particolare supporto digitale (disco magnetico, disco ottico, etc.), o da un particolare dispositivo o sistema di output (video, stampa);
% indipendenza dal software, sia sistemi operativi, sia applicazioni deputate alla creazione, analisi, manipolazione e visualizzazione di testi elettronici; (formati di dati proprietari mutamente incompatibili)
% indipendenza logica dalle applicazioni ovvero indipendenza semantica dello schema di codifica da un particolare processo applicativo.

% L’archiviazione su supporto digitale del patrimonio letterario e culturale delle culture mondiali deve misurarsi con questi problemi, e adottare degli schemi di codifica capaci di garantire la massima portabilità. 



% Una risorsa informativa digitale è portabile se è intercambiabile tra sistemi diversi, riutilizzabile in molteplici processi computazionali anche a distanza di tempo, e integrabile da ulteriori risorse informative omogenee


%  Esso deve divenire uno standard
% I vantaggi di uno standard formale o informale, oltre alla portabilità sta anche nella sua apertura, ovvero nella disponibilità pubblica delle sue specifiche.



% La disposizione alla rappresentazione di strutture astratte non pone limiti alla natura e tipologia delle caratteristiche testuali che si possono codificare in un testo elettronico. Queste possono essere utilizzate indifferentemente 

% Se il linguaggio è dotato di una sintassi che permette di specificare le relazioni tra gli elementi, essa può essere usata per rappresentare la struttura e l’organizzazione del testo a un determinato livello di descrizione, o i rapporti tra elementi appartenenti a diversi livelli.


% un sistema di codifica dichiarativa assiste un autore nel processo di scrittura, poiché focalizza l’attenzione sul contenuto di un testo (o sulla struttura del contenuto) piuttosto che sulla sua forma grafica.

% I sistemi di markup dichiarativo introducono consistenti vantaggi anche nei processi produttivi editoriali e nella gestione dei flussi informativi aziendali. Poiché un medesimo schema di codifica dichiarativo può essere utilizzato in molteplici forme di trattamento informatico, i costi di produzione e gestione di una base dati testuale vengono fortemente ridotti.

% I sistemi di codifica dichiarativa peraltro si prestano ottimamente per rappresentare strutture complesse come riferimenti incrociati e collegamenti tra elementi all’interno di un testo, ma anche tra più testi

% Infatti un database offre dei notevoli vantaggi dal punto di vista delle prestazioni computazionali e della velocità di ricerca, anche se richiede in generale una ingente quantità di memoria per l’archiviazione.

% La codifica permette allo studioso di esplicitare le sue ipotesi interpretative.

% OHCO: efficienza computazionale che la struttura gerarchica mostra; 

% I linguaggi di markup dichiarativi permettono di predicare l’appartenenza di un dato segmento testuale a una classe di strutture testuali definita dall’utente;
%  Così è possibile descrivere formalmente le caratteristiche di un testo in modo indipendente da particolari finalità di trattamento
% da contingenti forme di presentazione grafica su un qualsivoglia supporto fisico

% I linguaggi di markup dichiarativi, e in particolare SGML e XML, si sono rivelati dei veri e propri strumenti di supporto all’analisi computazionale dei testi
% la sintassi del linguaggio di codifica può essere usata per rappresentare le relazioni tra gli elementi strutturali di un testo, a un determinato livello di descrizione.

