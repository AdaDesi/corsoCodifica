% l’applicazione di procedure informatiche al trattamento dei testi richiede anche la simulazione dei processi che su di essi vengono effettuati.
% scrittura (il momento in cui il testo ha origine), edizione, lettura, analisi, interpretazione, archiviazione, catalogazione.

% Alcune caratteristiche sono comuni a tutti o a molti di questi tipi di testi, mentre altre sono assolutamente specifiche.


% Una rappresentazione codificata di un testo dunque è «vera» se è internamente coerente, accettabile razionalmente nell’ambito di una teoria, in grado di rappresentare i fenomeni testuali rilevanti nel contesto di quella teoria o prospettiva metodologica, ed eventualmente di rendere conto dei rapporti tra strutture e fenomeni emergenti da rappresentazioni (punti di vista) diverse.

% E come rileva lo stesso Dennet, possiamo benissimo sbagliarci nell’interpretare.

% La codifica elettronica di un testo, in quanto rappresentazione di un testo e delle sue caratteristiche mediante un linguaggio formale, si colloca interamente all’interno del processo analitico-interpretativo,

% L’utilizzazione delle tecnologie e delle metodologie informatiche ci impone di esplicitare il complesso di nozioni implicite nel dominio degli studi linguistici e letterari, e di sottoporle a verifica sperimentale.

% attività pratica è stata la continua riflessione circa i migliori metodi e strumenti formali per condurre il delicato compito di rappresentare quegli oggetti complessi, plurali e multiformi che sono i testi, soprattutto quelli che rientrano nella difficilmente definibile categoria dei testi letterari.

% Queste assunzioni a loro volta non sono individuali: esse sono condivise da una comunità di studiosi che hanno in comune metodologie e pratiche disciplinari, ontologie pratiche, criteri di accettabilità razionale, anche se possono divergere sulla interpretazione di particolari fenomeni.

% l’ambito in cui l’applicazione dell’informatica allo studio dei testi letterari ha avuto il maggiore sviluppo è l'analisi quantitativa della rappresentazione lineare del testo.

\begin{frame}
	\frametitle{Approfondimenti e Conclusioni}
	\framesubtitle{per comprendere la codifica}
	\addtocounter{nframe}{1}

	\begin{block}{Ricapitolando}
		Il trasferimento del testo su supporto informatico propone allo studioso una serie di quesiti teorici (oltre a numerosi problemi pratici) a partire dal momento della decisione su quale particolare oggetto del mondo sia da considerare come ``fonte'' della codifica e memorizzazione.
     \end{block}
     
   
\end{frame}

\begin{frame}
	\frametitle{Approfondimenti e Conclusioni}
	\framesubtitle{per comprendere la codifica}
	\addtocounter{nframe}{1}

	\begin{block}{In conclusione}
		Secondo la teoria OHCO un testo è una struttura gerarchica di oggetti logici, e la codifica non fa altro che esplicitare questa sua struttura essenziale.
     \end{block}
     
     \begin{block}{In conclusione}
		Individuazione di diversi unità o elementi di contenuto testuale a partire da un punto di vista interno alle discipline che studiano ed analizzano il testo.
	 \end{block}
   
\end{frame}


\begin{frame}
	\frametitle{Approfondimenti e Conclusioni}
	\framesubtitle{per comprendere la codifica}
	\addtocounter{nframe}{1}

	\begin{block}{Punti di vista}
		 TEI ha una struttura modulare, in cui ogni modulo corrisponde alla rappresentazione di un determinato punto di vista metodologico sul testo.
     \end{block}
     
     \begin{block}{TEI e Gerarchie sovrapposte}
		L’impianto finale dello schema di codifica della Text Encoding Initiative ha accolto in parte la nozione di sottoprospettiva ed ha sviluppato una serie di costrutti sintattici e semantici in grado di rappresentare adeguatamente fenomeni di sovrapposizione e di parallelismo tra elementi testuali in XML.
	 \end{block}
	 
   
\end{frame}

\begin{frame}
	\frametitle{Approfondimenti e Conclusioni}
	\framesubtitle{per comprendere la codifica}
	\addtocounter{nframe}{1}
     
     \begin{block}{TEI e Gerarchie sovrapposte}
		La TEI ha sviluppato un sistema di codifica abbastanza potente da permettere di rappresentare le caratteristiche testuali che interessano e di poter esplicitare le sue interpretazioni sul ruolo di tali caratteristiche.
	 \end{block}

\end{frame}

\begin{frame}
	\frametitle{Approfondimenti e Conclusioni}
	\framesubtitle{per comprendere la codifica}
	\addtocounter{nframe}{1}
     
     \begin{block}{Scopo applicativo}
		 Occorre tenere presente nella rappresentazione del testo anche i possibili processi applicativi a cui esso può essere sottoposto.
	 \end{block}

\end{frame}


\begin{frame}
	\frametitle{Approfondimenti e Conclusioni}
	\framesubtitle{per comprendere la codifica}
	\addtocounter{nframe}{1}
     
     \begin{block}{Scopo applicativo}
        In ultima analisi, la codifica informatica di un testo può essere vista come il prodotto di un insieme di inferenze che vengono espresse mediante un linguaggio formalizzato. (Ciotti)
	 \end{block}

\end{frame}

\begin{frame}
	\frametitle{Approfondimenti e Conclusioni}
	\framesubtitle{per comprendere la codifica}
	\addtocounter{nframe}{1}
     
     \begin{block}{Codifica del testo: riassumendo}
       \begin{itemize}
           \item elaborazione di un quadro teorico di riferimento entro cui collocare i procedimenti analitici
           \item definizione di un modello di rappresentazione informatica o codifica del testo e delle strutture rilevanti in relazione al con testo di riferimento
           \item individuazione di metodi e processi di analisi testuale applicabili al modello del testo e loro definizione sottoforma di procedure formali o algoritmi
           \item implementazione del modello di rappresentazione e dei processi di analisi mediante adeguati linguaggi informatici
           \item applicazione delle procedure informatiche al testo digitalizzato
           \item analisi e interpretazione critica dei risultati
       \end{itemize}
	 \end{block}

\end{frame}

% riassumendo:
%%;
%%;
%%;
%%;
%%;
%%.
% Si tratta di costruire un modello informatico del testo nel quadro di un determinato contesto teorico, per poi interrogare opportunamente tale modello e avanzare ipotesi interpretative sul testo.



% Dalle slide Roberto
%Non è codifica di un testo
%fare una scansione di un documento e diffonderne le
%immagini (ad es. in formato PDF) → digitalizzazione
%usare un software OCR per ricavarne una versione in
%formato ASCII (anche se si tratta di MRF)
%creare una pagina HTML sulla base di tale testo ASCII
%creare un documento Word ...
%creare un database testuale sulla base di tale testo
%ASCII


% Dalle slide Roberto
%Fare la codifica di un testo significa
%convertirlo in un formato comprensibile per
%l’elaboratore
%usare, a tal fine, un linguaggio di codifica di tipo
%formale
%definire e seguire uno schema di codifica ben preciso
%stabilito in base alle caratteristiche del testo che si
%intende esplicitare per il computer (modello di codifica)




%  quell’oltre si apre lo spazio dell’interpretazione. È questo spazio tra la sequenza discorsiva dei significanti e l’universo semantico e sintattico della narrazione a essere difficilmente valicabile.


% Se, invece, ci si propone di studiare i fenomeni testuali dell’intreccio dei campi semantici sui quali si basa la narrazione, solo un preventivo intervento interpretativo dello studioso può fornire i dati a un sistema informatico che sia in grado di analizzarli in modo quantitativo. (il calcolatore è bravo a fare conti).

% assumere la prospettiva di un’analisi del testo (letterario) assistita dal computer.

% Il fatto è che quel testo elettronico rende concreto semplicemente uno dei modelli possibili 
% L’operazione di codifica resta dunque un’opera d’interpretazione
%  Sarà possibile, in tal modo, instaurare una sorta di rapporto dialettico tra testo, dati scaturiti dall’elaborazione informatica e ipotesi dello studioso, che realmente potrebbe aiutarci a disegnare un profilo del tutto nuovo dell’operazione di critica testuale. Per arrivare, infine, alla realizzazione di un modello attendibile e utile del documento da studiare, stabilendone e dichiarandone i livelli di rappresentazione

% Limiti e difetti dell'XML


% Crediamo che una adeguata soluzione pragmatica di questi problemi sia da individuare nell’associazione di un modello implementato da un markup language, fin dove è possibile, e di un modello grafico del documento, implementato in uno dei formati grafici standard, tra quelli sviluppati nell’ambito delle tecnologie di computer grafica (immagini facsimile).

% Una tale forma di modellizzazione informatica di un documento testuale risulterebbe, peraltro, la più adeguata nel caso specifico dei manoscritti, per i quali nessuna descrizione di tipo linguistico sarebbe in grado di rappresentare tutte le informazioni visuali che una immagine digitale è in grado di veicolare.



% A questa struttura rigida del codice nell’ambito informatico (codice binario) va confrontata la complessa intersezione di codici che costituiscono un testo (da memorizzare).



%   Il computer può fornire una grande massa di analisi quantitative e statistiche: con la produzione di spogli lessicali, indici di frequenze, concordanze, attività che un elaboratore elettronico svolge con una efficienza di gran lunga superiore al più volenteroso studioso, e con una estensione pressoché illimitata.