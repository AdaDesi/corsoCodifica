% l’applicazione di procedure informatiche al trattamento dei testi richiede anche la simulazione dei processi che su di essi vengono effettuati.
% scrittura (il momento in cui il testo ha origine), edizione, lettura, analisi, interpretazione, archiviazione, catalogazione.

% Alcune caratteristiche sono comuni a tutti o a molti di questi tipi di testi, mentre altre sono assolutamente specifiche.

% In conclusione secondo la teoria OHCO un testo è una struttura gerarchica di oggetti logici, e la codifica non fa altro che esplicitare questa sua struttura essenziale.

% Ci accorgiamo dunque che la pratica comune nelle discipline che stu- diano i testi è quella di definire il loro oggetto a partire da un punto di vista interno alla disciplina stessa. Ciò dà luogo alla individuazione di diversi unità o elementi di conte- nuto testuale.

% TEI ha una struttura modulare, in cui ogni modulo corrisponde alla rappresentazione di un determinato punto di vi- sta metodologico sul testo.

% L’impianto finale dello schema di codifica della Text Encoding Initiative ha accolto in parte la nozione di sottoprospettiva ed ha svi- luppato una serie di costrutti sintattici e semantici in grado di rappre- sentare adeguatamente fenomeni di sovrapposizione e di parallelismo tra elementi testuali in XML.

% Una rappresentazione codificata di un testo dunque è «vera» se è internamente coerente, accettabile razionalmente nell’ambito di una teoria, in grado di rappresentare i fenomeni testuali rilevanti nel conte- sto di quella teoria o prospettiva metodologica, ed eventualmente di rendere conto dei rapporti tra strutture e fenomeni emergenti da rap- presentazioni (punti di vista) diverse.

% Il trasferimento del testo su supporto informatico propone allo studioso una serie di quesiti teorici (oltre a numerosi problemi pratici) a partire dal momento della decisione su quale particolare oggetto del mondo sia da considerare come “fonte” della memorizzazione.

% E come rileva lo stesso Dennet, possiamo benissimo sbagliarci nell’interpretare.

% La codifica elettronica di un testo, in quanto rappresentazione di un testo e delle sue caratteristiche median- te un linguaggio formale, si colloca interamente all’interno del proces- so analitico-interpretativo,

% si tratta quindi di individuare o sviluppare un sistema di codifica abbastanza potente da permettere a ogni studio- so, da qualsiasi punto di vista disciplinare, di rappresentare le caratte- ristiche testuali che lo interessano e di poter esplicitare le sue interpre- tazioni sul ruolo di tali caratteristiche.

% Occorre dunque tenere presente nella rappresentazione del testo anche i possibili processi ap- plicativi a cui esso può essere sottoposto.

% Queste assunzioni a loro volta non sono individuali: esse sono condivise da una comunità di studiosi che hanno in comune metodologie e pratiche disciplinari, ontologie pratiche, criteri di accettabilità razionale, anche se possono divergere sulla interpretazione di particolari fenomeni.

% attività pratica è stata la continua riflessione circa i migliori metodi e strumenti formali per condurre il delicato compito di rappresentare quegli oggetti complessi, plurali e multiformi che sono i testi, soprattutto quelli che rientrano nella difficilmente definibile categoria dei testi letterari.

% L’utilizzazione delle tecnologie e delle metodologie informatiche ci impone di esplicitare il complesso di nozioni implicite nel dominio degli studi linguistici e letterari, e di sottoporle a verifica sperimentale.
%  quell’oltre si apre lo spazio dell’interpretazione. È questo spazio tra la sequenza discorsiva dei significanti e l’universo semantico e sintattico della narrazione a essere difficilmen- te valicabile.
% l’ambito in cui l’applicazione dell’informatica allo studio dei testi letterari ha avuto il maggiore sviluppo è l'analisi quantitativa della rappresentazione lineare del testo.

% Se, invece, ci si propone di studiare i fenomeni testuali dell’intreccio dei campi semantici sui quali si basa la narrazione, solo un preventivo intervento interpretativo dello studioso può fornire i dati a un sistema informatico che sia in grado di analizzarli in modo quantitativo. (il calcolatore è bravo a fare conti).
% assumere la prospettiva di un’analisi del testo (letterario) as- sistita dal computer.
% riassumento:
%%elaborazione di un quadro teorico di riferimento entro cui col- locare i procedimenti analitici;
%%definizione di un modello di rappresentazione informatica o codifica del testo e delle strutture rilevanti in relazione al con- testo di riferimento;
%%individuazione di metodi e processi di analisi testuale applica- bili al modello del testo e loro definizione sottoforma di proce- dure formali o algoritmi;
%%implementazione del modello di rappresentazione e dei proces- si di analisi mediante adeguati linguaggi informatici;
%%applicazione delle procedure informatiche al testo digitalizza- to;
%%analisi e interpretazione critica dei risultati.
% Si tratta di costruire un modello informatico del testo nel quadro di un determinato contesto teorico, per poi interrogare oppor- tunamente tale modello e avanzare ipotesi interpretative sul testo.

% Il fatto è che quel testo elettronico rende concreto semplicemente uno dei modelli possibili 
% L’operazione di codifica resta dunque un’opera d’interpretazione
%  Sarà possibile, in tal modo, instaurare una sorta di rapporto dialettico tra testo, dati scaturiti dall’elaborazione informatica e ipotesi dello studioso, che re- almente potrebbe aiutarci a disegnare un profilo del tutto nuovo dell’operazione di critica testuale. Per arrivare, infine, alla realizzazione di un mo- dello attendibile e utile del documento da studiare, stabilendone e di- chiarandone i livelli di rappresentazione

% Limiti e difetti dell'XML


% Crediamo che una adeguata soluzione pragmatica di questi problemi sia da individuare nel- l’associazione di un modello implementato da un markup language, fin dove è possibile, e di un modello grafico del documento, implementato in uno dei formati grafici standard, tra quelli sviluppati nell’ambito delle tecnologie di computer grafica (immagini facsimile).

% Una tale forma di modellizzazione informatica di un documento testuale risulterebbe, peraltro, la più adeguata nel caso specifico dei manoscritti, per i quali nessuna descrizione di tipo linguistico sarebbe in grado di rappresentare tutte le informazioni visuali che una immagine digitale è in grado di veicolare.

% In ultima analisi, la codifica informatica di un testo può essere vista come il prodotto di un insieme di inferenze che vengono espresse mediante un linguaggio formalizzato. (ciotti)