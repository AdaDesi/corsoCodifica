% l’applicazione di procedure informatiche al trattamento dei testi richiede anche la simulazione dei processi che su di essi vengono effettuati.
% scrittura (il momento in cui il testo ha origine), edizione, lettura, analisi, interpretazione, archiviazione, catalogazione.

% Alcune caratteristiche sono comuni a tutti o a molti di questi tipi di testi, mentre altre sono assolutamente specifiche.

% In conclusione secondo la teoria OHCO un testo è una struttura gerarchica di oggetti logici, e la codifica non fa altro che esplicitare questa sua struttura essenziale.

% Ci accorgiamo dunque che la pratica comune nelle discipline che stu- diano i testi è quella di definire il loro oggetto a partire da un punto di vista interno alla disciplina stessa. Ciò dà luogo alla individuazione di diversi unità o elementi di conte- nuto testuale.

% TEI ha una struttura modulare, in cui ogni modulo corrisponde alla rappresentazione di un determinato punto di vi- sta metodologico sul testo.

% L’impianto finale dello schema di codifica della Text Encoding Initiative ha accolto in parte la nozione di sottoprospettiva ed ha svi- luppato una serie di costrutti sintattici e semantici in grado di rappre- sentare adeguatamente fenomeni di sovrapposizione e di parallelismo tra elementi testuali in XML.

% Una rappresentazione codificata di un testo dunque è «vera» se è internamente coerente, accettabile razionalmente nell’ambito di una teoria, in grado di rappresentare i fenomeni testuali rilevanti nel conte- sto di quella teoria o prospettiva metodologica, ed eventualmente di rendere conto dei rapporti tra strutture e fenomeni emergenti da rap- presentazioni (punti di vista) diverse.

% Il trasferimento del testo su supporto informatico propone allo studioso una serie di quesiti teorici (oltre a numerosi problemi pratici) a partire dal momento della decisione su quale particolare oggetto del mondo sia da considerare come “fonte” della memorizzazione.

% E come rileva lo stesso Dennet, possiamo benissimo sbagliarci nell’interpretare.

% La codifica elettronica di un testo, in quanto rappresentazione di un testo e delle sue caratteristiche median- te un linguaggio formale, si colloca interamente all’interno del proces- so analitico-interpretativo,

% si tratta quindi di individuare o sviluppare un sistema di codifica abbastanza potente da permettere a ogni studio- so, da qualsiasi punto di vista disciplinare, di rappresentare le caratte- ristiche testuali che lo interessano e di poter esplicitare le sue interpre- tazioni sul ruolo di tali caratteristiche.

% Occorre dunque tenere presente nella rappresentazione del testo anche i possibili processi ap- plicativi a cui esso può essere sottoposto.

% Queste assunzioni a loro volta non sono individuali: esse sono condivise da una comunità di studiosi che hanno in comune metodologie e pratiche disciplinari, ontologie pratiche, criteri di accettabilità razionale, anche se possono divergere sulla interpretazione di particolari fenomeni.

% attività pratica è stata la continua riflessione circa i migliori metodi e strumenti formali per condurre il delicato compito di rappresentare quegli oggetti complessi, plurali e multiformi che sono i testi, soprattutto quelli che rientrano nella difficilmente definibile categoria dei testi letterari.

% L’utilizzazione delle tecnologie e delle metodologie informatiche ci impone di esplicitare il complesso di nozioni implicite nel dominio degli studi linguistici e letterari, e di sottoporle a verifica sperimentale.
%  quell’oltre si apre lo spazio dell’interpretazione. È questo spazio tra la sequenza discorsiva dei significanti e l’universo semantico e sintattico della narrazione a essere difficilmen- te valicabile.