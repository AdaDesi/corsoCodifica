% analizzare alcuni problemi teorici impli- cati dalla codifica informatica dei testi.
% La codifica infatti è un processo assai più complesso delle semplice e meccanica correlazione biunivoca di strutture rappresentazionali.
% problema del testo: la determinazione di «cosa sia un testo» e di quale legame sussista tra questa determinazione, i processi dell’interpretazione e i linguaggi con i quali essa viene enunciata.
%  uno di carattere epistemologico, riguarda la natu- ra della codifica come processo di rappresentazione.
% carattere ontologico, e concerne il concet- to generale di testo che «emerge» dalle teorie dei sistemi di codifica.

% domanda: codifica è un processo interpretativo oppure un processo riproduttivo?
%% lo schema di codifica TEI impone al responsabile della codifica di effettuare delle scelte teoriche e interpretative che non sono pertinenti alla sua opera di semplice trascrittore.

% La distinzione tra sistemi procedurali e sistemi dichiarativi non è parallela ma orto- gonale a quella tra codifica presentazionale e codifica analitica o strut- turale.

% Naturalmente questo è possibile se tale descrizione del supporto fi- sico di un testo è riducibile a un struttura gerarchica.

% I problemi e le difficoltà determinati dagli schemi SGML per una codifica presenta- zionale in effetti, sono determinati proprio da questa metastruttura

% indagare più a fondo la natura della codifica e dell’idea di testo che la codifica presuppone.

% Si può infatti veramente sostenere che la codifica sia un semplice processo di trascrizione?

% la rappresentazione informatica è un processo semiotico: Ogni atto rappresentazionale o semiotico implica dei processi interpretativi 

% Conseguentemente sosteniamo che ogni processo di codifica (inclusi quelli di cui ci occupiamo in questa sede) è il risultato di una interpretazione.

%% esempio trascrizione caratteri: l’assunzione che una data traccia grafica «A» sia un token di un data classe astratta di tracce che identifichiamo come il carattere «a». Richiesti molti sforzi interpretativi

% in linea di principio non è sempre possibile predicare in modo non ambiguo l’appartenenza di una certa «traccia» su un supporto testuale fisico a una certa classe di iscrizioni che chiamiamo «carattere»

% problema: utilizzazione dei simboli del linguaggio informatico in funzione di representamen dei caratteri alfanumerici del testo

%% Ogni interpretazione può godere di diversi gradi di certezza e di sog- gettività. In ogni caso non esiste nessun genere di rappresentazione di un testo che si possa definire libera da processi interpretativi.

% lo schema di codifica è un linguaggio teorico usato per costruire teorie o modelli di fenomeni testuali.

% la stessa “costruzione” di un linguaggio teorico ri- flette un determinato modello del mondo.

% il linguaggio di codifica informatica del testo, implica una teoria ontologica del testo

% obiettivo: 1) sviluppare teorie e modelli formali del testo (o di alcuni suoi livelli descrittivi)
2) individuare formalismi atti a esprimerli in modo computazio- nalmente accettabile

% OHCO: La ragione di tanto attaccamento all’idea di struttura gerarchica ovviamente non è immotivata. Il fatto è che XML (e SGML) può esse- re considerato sia un formalismo sia un modello di dati espresso da quel formalismo, e tale (meta)modello è appunto un albero ordinato etichettato.
% Il prezzo costituito dall’adozione di un modello di dati così vinco- lante, d’altra parte, paga il vantaggio di potere validare in modo auto- matico ogni istanza di dati rispetto al modello mediante algoritmi ge- nerali ben conosciuti e computazionalmente trattabili, ciò che a sua volta consente di costruire sistemi di elaborazione degli stessi dati consistenti ed efficaci (al netto dei costrutti ID/IDREF)

% Le manifestazioni di queste difficoltà sono state comu- nemente rubricate come il problema delle gerarchie sovrapposte (over- lapping hierarchies, OH d’ora in poi)
% In termini semplici il problema OH dal punto di vista sintattico consiste nel fatto che, dati due oggetti logici presenti in un testo, le coppie di tag bilanciati che li rappresentano non si annidano propriamente ma si sovrappongono.
%Tale situazione è sintatticamente e semanticamente vietata in XML