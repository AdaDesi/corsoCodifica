%slide relative ad introdurre gli elementi di XML Schema Definition (XSD) 

% Description and validation of the XML elements and attributes are written in XSD using element declarations and attribute declarations.

%  XSD is a grammar language


% declaring elements and attributes, adding child elements, defining occurrence and order of child elements
% XSD data types and see how to perform basic data validation using XSD.

% Writing the XSD code for describing and validating an element in an XML document is called element declaration.

%% Element declaration
% The element declaration we created above has an attribute named name.
% This attribute specifies the name of the element expected in the XML instance document
% esistono altri attributi dell'elemento xsd per la dichiarazione di elementi

%% Attribute declaration
% The XSD code to describe and validate an attribute in an XML instance document is called attribute declaration.
% An attribute cannot exist without a parent element;

%% Esempio element declaration più attribute declaration
%  <xsd:schema xmlns:xsd="http://www.w3.org/2001/XMLSchema">
%   <xsd:element name="Employee">
%     <xsd:complexType>
%       <xsd:attribute name="Name"/>
%     </xsd:complexType>
%   </xsd:element>
% </xsd:schema>

%% Simple Types
%%Elements that do not have child elements or attributes are simple type elements.

%% Complex Type


%frame 01
\begin{frame}
    \frametitle{Elementi per la definizione degli schemi xml}
    \framesubtitle{principi XML Schema Definition}
    \addtocounter{nframe}{1}

    \begin{block}{Cos'è uno schema XML}
        Uno schema XML è un documento XML standard che descrive come deve essere realizzato un altro documento XML.
        \\ Ci riferiamo a questa tecnologia con l'acronimo XSD.
    \end{block}
    
    \begin{block}{A cosa serve uno Schema XML}
     I documenti XSD sono usati per validare documenti XML. 
    \\ Tuttavia un documento XSD viene realizzato tramite l'uso di un vocabolario predefinito riferibile attraverso un namespave ad un URI standard.
    \end{block}

\end{frame}

%frame 02
\begin{frame}
    \frametitle{Elementi per la definizione degli schemi xml}
    \framesubtitle{principi XSD}
    \addtocounter{nframe}{1}

    \begin{block}{XSD Schema}
        Il termine XSD o XML Schema denota un documento XML che descrive e valida la struttura e il contenuto di un altro documento XML.
    \end{block}

    \begin{block}{XSD Schema}
        \textbf{Dichiarazione del documento (declaration) e istanza del documento (instance).}
    \end{block}
    
\end{frame}

%frame 03
\begin{frame}
    \frametitle{Elementi per la definizione degli schemi xml}
    \framesubtitle{principi XSD}
    \addtocounter{nframe}{1}

    \begin{block}{XSD elemento root}
        L'elemento radice di uno schema XSD è sempre l'elemento ``\texttt{<schema>}''. 
        \\ Tutte le definizione devono seguire quindi l'elemento ``\texttt{<schema>}''.
    \end{block}

    \begin{block}{XSD Schema}
        Tutti gli elementi e gli attributi dello schema sono dichiarati all'interno del namespace ``\texttt{http://www.w3.org/2001/XMLSchema.}''. 
        \\ Tutti i documenti XSD contengono la dichiarazione a questo namespace con prefisso convenzionale \textbf{xsd} oppoure \textbf{xs}.
    \end{block}
    
\end{frame}

\begin{frame}
    \frametitle{Elementi per la definizione degli schemi xml}
    \framesubtitle{principi XSD}
    \addtocounter{nframe}{1}

    \begin{block}{XSD componenti di base}
       I componenti di base di uno Schema XSD sono le dichiarazioni degli elementi e le dichiarazioni degli attributi.
    \end{block}

    \begin{block}{XSD Schema}
        \textbf{Le dichiarazioni più complesse si poggiano su queste unità: elementi e attributi.}    
    \end{block}
    
\end{frame}

\begin{frame}
    \frametitle{Elementi per la definizione degli schemi xml}
    \framesubtitle{principi XSD}
    \addtocounter{nframe}{1}

    \begin{block}{XSD dichiarazioni}
       Scrivere un pezzo di codice XSD per descrivere e validare un elemento per un documento XML è detto \textit{element declaration}.
    \end{block}

    \begin{block}{XSD dichiarazioni di base}
        XSD permette di dichiarare elementi, attributi e di spcificare il numero di figli, le occorrenze, l'ordine di apparizione, e i tipi di dati del content model.
    \end{block}
    
\end{frame}


\begin{frame}
    \frametitle{Elementi per la definizione degli schemi xml}
    \framesubtitle{principi XSD}
    \addtocounter{nframe}{1}

    \begin{block}{Element Types: simple and complex}
        La dichiarazione di un elemento può avere un tipo semplice (simple type) oppure un tipo complesso (comples type) a seconda della sua struttura e del suo contenuto.


    \end{block}

    \begin{block}{Simple Type e Complex Type}
        La dichiarazione di un elemento ha un tipo semplice se non possiede \textbf{né figli né attributi}.
        \\ La dichiarazione di un elemento ha un tipo complesso in tutti gli altri casi.
    \end{block}
    
\end{frame}


%frame 04
\begin{frame}
    \frametitle{Elementi per la definizione degli schemi xml}
    \framesubtitle{principi XSD}
    \addtocounter{nframe}{1}

    %\defverbatim{\xsdfirst}{%
    %\begin{tiny}
    %\begin{verbatim}
    %    <xsd:schema xmlns:xsd="http://www.w3.org/2001/XMLSchema">
    %        <xsd:element name="text"/>
    %    </xsd:schema>
    %\end{verbatim}
%    \end{tiny}
 %   }

  %  \defverbatim{\xmlfirst}{%
   % \begin{tiny}
    %\begin{verbatim}
    %         <text>Il primo documento XML Validato</text>
   % \end{verbatim}
   % \end{tiny}
   % }


    \begin{block}{XSD esempio}
       % {\xsdfirst}
\texttt{<xsd:schema xmlns:xsd=``http://www.w3.org/2001/XMLSchema''>}
            \\\texttt{<xsd:element name=``text''/>}
        \\\texttt{</xsd:schema>}
    \end{block}

    \begin{block}{XSD esempio elemento di tipo semplice}
        % {\xsdfirst}
 \texttt{<text>Il primo documento XML Validato</text>}
     \end{block}


    
\end{frame}

\begin{frame}
    \frametitle{Elementi per la definizione degli schemi xml}
    \framesubtitle{principi XSD}
    \addtocounter{nframe}{1}

    \begin{block}{XML XSD esempio}
        Il documento XML istanza dello schema XSD per essere valido deve contenere un elemento radice.
        Validare il documento XML con il relativo XSD con XMLlint.
        
    \end{block}

    \begin{block}{XMLlint}
        \texttt{xmllint xmlfirst.xml --schema ../schema/xsd/xsdfirst.xsd}
    \end{block}
    
    
\end{frame}

\begin{frame}
    \frametitle{Elementi per la definizione degli schemi xml}
    \framesubtitle{principi XSD}
    \addtocounter{nframe}{1}

    \begin{block}{Element Complex Types: esempio}
         
        %If an element contains child elements or attributes, it has a complex type.
        \texttt{
            <xsd:schema xmlns:xsd=``http://www.w3.org/2001/XMLSchema''>
            <xsd:element name=``Employee''>
                <xsd:complexType>
                    <xsd:attribute name=``FirstName''/>
                </xsd:complexType>
            </xsd:element>
            </xsd:schema>
        }
    \end{block}

    \begin{block}{Element Complex Types: esempio}
         Il documento XML istanza dello schema: 
         \\\texttt{<Employee FirstName="Jacob"/>}
    \end{block}
    
\end{frame}
