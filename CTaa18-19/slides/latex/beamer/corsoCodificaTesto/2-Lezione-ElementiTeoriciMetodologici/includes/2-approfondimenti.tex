% l’applicazione di metodologie computazionali nell’ambito della ricerca umanistica comporta due tipi, o meglio due fasi di formalizzazione:
% definizione e implementazione di strutture dati adeguate alla cattura dei fenomeni di interesse dell’umanista, e in particolare alla rappresentazione formale dei testi;
% specificazione di algoritmi che, applicati alle strutture dati, siano in grado di simulare i processi di manipolazione dei testi tipici della ricerca umanistica o in generale delle pratiche sociali che hanno a che fare in vario modo con i testi.

%% lo schema di codifica TEI impone al responsabile della codifica di effettuare delle scelte teoriche e interpretative che non sono pertinenti alla sua opera di semplice trascrittore.

%  uno di carattere epistemologico, riguarda la natura della codifica come processo di rappresentazione.
% carattere ontologico, e concerne il concetto generale di testo che «emerge» dalle teorie dei sistemi di codifica.

% domanda: codifica è un processo interpretativo oppure un processo riproduttivo?
%% lo schema di codifica TEI impone al responsabile della codifica di effettuare delle scelte teoriche e interpretative che non sono pertinenti alla sua opera di semplice trascrittore.


% la rappresentazione informatica è un processo semiotico: Ogni atto rappresentazionale o semiotico implica dei processi interpretativi 

% indagare più a fondo la natura della codifica e dell’idea di testo che la codifica presuppone.


% Naturalmente questo è possibile se tale descrizione del supporto fisico di un testo è riducibile a un struttura gerarchica.

% I problemi e le difficoltà determinati dagli schemi SGML per una codifica presentazionale in effetti, sono determinati proprio da questa metastruttura

% problema: utilizzazione dei simboli del linguaggio informatico in funzione di representare dei caratteri alfanumerici del testo

\begin{frame}
	\frametitle{Approfondimenti e Conclusioni}
	\framesubtitle{per comprendere la codifica}
	\addtocounter{nframe}{1}

	\begin{block}{Codifica del testo}
		L’applicazione di metodologie computazionali nell’ambito della ricerca umanistica comporta due aspetti di formalizzazione
	\end{block}

	\begin{block}{Due elementi}
		Formalizzazione dei dati e formalizzazione dell'elaborazione
	\end{block}

\end{frame}

\begin{frame}
	\frametitle{Approfondimenti e Conclusioni}
	\framesubtitle{per comprendere la codifica}
	\addtocounter{nframe}{1}

	\begin{block}{Formalizzazione dei dati}
		Definizione e implementazione di strutture dati adeguate alla cattura dei fenomeni di interesse dell’umanista, e in particolare alla rappresentazione formale dei testi;
	\end{block}

	\begin{block}{Formalizzazione dell'elaborazione}
		Specificazione di algoritmi che, applicati alle strutture dati, siano in grado di simulare i processi di manipolazione dei testi tipici della ricerca umanistica o in generale delle pratiche sociali che hanno a che fare in vario modo con i testi.
	\end{block}

\end{frame}


\begin{frame}
	\frametitle{Approfondimenti e Conclusioni}
	\framesubtitle{per comprendere la codifica}
	\addtocounter{nframe}{1}

	\begin{block}{Codifica del testo}
		Il problema della codifica testuale rientra in generale nel primo tipo di formalizzazione (Dati).
	\end{block}

\end{frame}

\begin{frame}
	\frametitle{Approfondimenti e Conclusioni}
	\framesubtitle{per comprendere la codifica}
	\addtocounter{nframe}{1}

	\begin{block}{Problemi teorici}
		La codifica è un processo assai più complesso della semplice e meccanica correlazione biunivoca di strutture rappresentazionali.
	\end{block}

\end{frame}

\begin{frame}
	\frametitle{Approfondimenti e Conclusioni}
	\framesubtitle{per comprendere la codifica}
	\addtocounter{nframe}{1}

	\begin{block}{Problema del testo}
		La specificazione di cosa sia un testo e di quale legame sussista tra questa specificazione, i processi dell’interpretazione e i linguaggi formali con i quali essa viene descritta è tutt'altro che banale.
	\end{block}

\end{frame}

\begin{frame}
	\frametitle{Approfondimenti e Conclusioni}
	\framesubtitle{per comprendere la codifica}
	\addtocounter{nframe}{1}

	\begin{block}{Norme TEI}
		Le linee guida di codifica TEI impongono a chi codifica di effettuare delle scelte teoriche e interpretative che non sono imputabili alla semplice trascrizione.
    \end{block}
    
    \begin{block}{Processo di codifica}
        La codifica è un processo interpretativo non solo un processo riproduttivo.
        \\Non è quindi un semplice processo di trascrizione!
    \end{block}
    

\end{frame}


\begin{frame}
	\frametitle{Approfondimenti e Conclusioni}
	\framesubtitle{per comprendere la codifica}
	\addtocounter{nframe}{1}

	\begin{block}{Codifica come interpretazione}
		Conseguentemente ogni processo di codifica (inclusi quelli di codifica informatica del testo) è il risultato di una interpretazione da esplicitare e formalizzare.
    \end{block}
   
\end{frame}

\begin{frame}
	\frametitle{Approfondimenti e Conclusioni}
	\framesubtitle{per comprendere la codifica}
	\addtocounter{nframe}{1}

	\begin{block}{Esempio}
		L’assunzione che una data traccia grafica ``A'' sia una istanza di una data classe astratta di tracce che identifichiamo come il carattere ``a'' nella sua forma maiuscola richiede molti sforzi interpretativi e quindi cognitivi.
    \end{block}
    
    \begin{block}{Rappresentazione e interpretazione}
        In linea di principio non è sempre possibile dire in modo non ambiguo che una traccia su un supporto fisico appartiene a una certa classe di iscrizioni che chiamiamo carattere.
    \end{block}

\end{frame}


\begin{frame}
	\frametitle{Approfondimenti e Conclusioni}
	\framesubtitle{per comprendere la codifica}
	\addtocounter{nframe}{1}

	\begin{block}{Certezza e soggettività}
		\textbf{Ogni interpretazione può godere di diversi gradi di certezza e di soggettività. In ogni caso non esiste nessun genere di rappresentazione di un testo che si possa definire libera da processi interpretativi.}
    \end{block}
   
\end{frame}

\begin{frame}
	\frametitle{Approfondimenti e Conclusioni}
	\framesubtitle{per comprendere la codifica}
	\addtocounter{nframe}{1}

	\begin{block}{Linguaggio teorico}
		Lo schema di codifica è un linguaggio teorico usato per costruire teorie o modelli di fenomeni testuali
    \end{block}

    \begin{block}{Linguaggio teorico}
        La stessa costruzione di un linguaggio teorico riflette un determinato modello del mondo (soggettivo o condiviso).
        \\ \textit{Si presuppone una teoria ontologica del testo}.
    \end{block}
   
\end{frame}

\begin{frame}
	\frametitle{Approfondimenti e Conclusioni}
	\framesubtitle{per comprendere la codifica}
	\addtocounter{nframe}{1}

	\begin{block}{Obiettivo}
		\begin{itemize}
			\item Sviluppare teorie e modelli formali del testo (o di alcuni suoi livelli descrittivi)
			\item Individuare formalismi atti a esprimerli in modo computazionalmente accettabile
		\end{itemize}
	
    \end{block}
   
\end{frame}

\begin{frame}
	\frametitle{Approfondimenti e Conclusioni}
	\framesubtitle{per comprendere la codifica}
	\addtocounter{nframe}{1}

	\begin{block}{OHCO}
		 Il modello di struttura gerarchica del testo si sposa perfettamente con XML, il quale può essere considerato sia un formalismo sia un modello di dati espresso da quel formalismo, e tale meta-modello è appunto un albero ordinato etichettato.
	
    \end{block}
   
\end{frame}

\begin{frame}
	\frametitle{Approfondimenti e Conclusioni}
	\framesubtitle{per comprendere la codifica}
	\addtocounter{nframe}{1}

	\begin{block}{OHCO: modello di dati così vincolante}
		Il modello OHCO presenta dei vincoli, che vanno sotto il nome di ``problema delle gerarchie sovrapposte'' (overlapping hierarchies).
	 \end{block}
	 
	 \begin{block}{Gerarchie Sovrapposte}
		Dati due oggetti logici presenti in un testo, le coppie di tag bilanciati che li rappresentano non si annidano propriamente ma si sovrappongono. \textit{Tale situazione è sintatticamente e semanticamente vietata in XML}.
 	\end{block}
   
\end{frame}

\begin{frame}
	\frametitle{Approfondimenti e Conclusioni}
	\framesubtitle{per comprendere la codifica}
	\addtocounter{nframe}{1}

	\begin{block}{Modelli gerarchici}
		Vantaggio di poter validare in modo automatico ogni istanza di dati rispetto al modello mediante algoritmi generali ben conosciuti e computazionalmente trattabili, ciò che a sua volta consente di costruire sistemi di elaborazione degli stessi dati consistenti ed efficaci (al netto dei costrutti ID/IDREF).
	 \end{block}
	 
   
\end{frame}

\begin{frame}
	\frametitle{Approfondimenti e Conclusioni}
	\framesubtitle{per comprendere la codifica}
	\addtocounter{nframe}{1}

	\begin{block}{Modelli non gerarchici}
		Negli ultimi anni la ricerca teorica sullo sviluppo di sistemi di markup non gerarchici ha avuto un notevole stimolo
	 \end{block}

	 \begin{block}{Modelli non gerarchici}
		Tutte le soluzioni proposte finora hanno grossi limiti per quel che concerne la facilità di gestione e manutenzione
	 \end{block}
	 
   
\end{frame}


% ipotetiche soluzioni
% Non è un caso che negli ultimi dieci anni, proprio in parallelo con l’inarrestabile diffusione di XML nel mondo dell’elaborazione testuale (e non solo) e della TEI nella comunità umanistica si sono moltiplicati i tentativi di trovare delle soluzioni definitive al problema.
% soluzioni interne e soluzioni esterne al paradigma XML. Per questo alla completezza e congruenza va affiancata una serie di ulteriori criteri valutativi di natura teorica, tecnica e pragmatica


% La visione pluralista del testo portata alle sue estreme conseguenze, eccede i limiti sintattici di un formalismo di codifica come XML. Lo standard, infatti, non è dotato di costrutti sintattici adeguati alla rappresentazione di molteplici sottoprospettive gerarchiche concorrenti che si sovrappongono ma che possono anche collegarsi e interrelarsi.

% Ogni modello descrive le caratteristiche del testo a un determinato livello, in base al punto di vista dell’osservatore, ma non coincide con esse.

% esempio con codice XML e grafico/immagine

% problema della complessità testuale
%% Se esistono proprietà dei testi irriducibili a qualsiasi formalizzazione anche minimale, allora queste non possono per definizione essere rappresentate e trattate con metodi computazionali.
% nella categoria delle gerarchie sovrapposte si possono distinguere diversi sottoproblemi di complessità crescente
% distinguere una “gerarchia delle gerarchie sovrapposte”.
% 1. Il caso più semplice e comune è quello della compresenza di due o più strutture (livelli) gerarchiche i cui elementi si sovrappongono.
% 2. concettualmente più complicato è quello di un elemento appartenente a una gerarchia che si estende oltre i confini dell’elemento in cui inizia o persino di uno dei suoi predecessori
% 3. Tecnicamente simile ma concettualmente distinto il caso di elementi composti da segmenti discontinui e non contigui 
% 4. Infine il caso più complesso che si verifica quando un dato elemento può auto-sovrapporsi illimitatamente.


% Leggibilità e facilità di comprensione da parte di un utente umano (sono esclusi dunque tutti i formati binari)
% Facilità di manutenzione e modifica
% Disponibilità di implementazioni software
% Compatibilità sintattica con XML
% Facilità di validazione (eventualmente sopravanzando le capacità di validazione di un parser XML standard)
% Possibilità di validazione incrociata tra gerarchie diverse Possibilità di formattazione ed elaborazione grafica e presentazionale
% Possibilità di estrapolare molteplici viste basate su uno o più tra le gerarchie presenti
% Possibilità di estrapolare sottoinsiemi gerarchici delle caratteri- stiche testuali
% Continuità del contenuto testuale serializzato

% Soluzioni interne al paradigma XML rientrano artifici sintattici che mantengono la conformità a XML
% XML come puro formalismo di serializzazione per modelli di dati non gerarchici
% Segmentazione: Un elemento logico che si sovrappone ai confini di un’altro (o di più altri) viene diviso in due (o n) elementi XML dello stesso tipo correlati mediante apposti attributi.
% Questa soluzione permetterebbe di risolvere sovrapposizioni di tipo (1) , (2) e (3), e può essere parzialmente validata mediante un oculato uso di attributi ID/IDREF.
% Elementi di congiunzione: Consiste nell’introduzione di un elemento XML (elemento di congiunzione) con la funzione metatestuale di esprimere l’unità logica di un fenomeno testuale rappresentato da più elementi XML distinti. Questa tecnica consente di trattare in linea teorica ogni caso di sovrapposizione e di non contiguità, di ordinamento inverso e di relazione n-aria tra oggetti testuali. Questo ultimo problema può essere risolto parzialmente adottando gli schemi di puntamento basati su range di caratteri previsto nello standard XPointer ma in questo caso l’elaborazione dei riferimenti richiederebbe l’uso di software ad hoc.
% Markup esterno (stand-off markup): Le tecniche basate su markup esterno sono di fatto identiche a quelle basate su elementi join. La differenza consiste nel fatto che in questo caso gli elementi che esprimono il collegamento di segmenti testuali nel documento XML base sono in un documento XML esterno. Il fatto di poter collocare i collegamenti in un documento autonomo consente di adottare un linguaggio XML per descrivere una struttura principale del documento base e uno diverso per la rappresentazione dei livelli di descrizione ulteriori.

% Ad esempio si può usare la TEI per la codifica del documento e XTM (XML Topic Maps) o RDF per esprimere le relazioni tra gli oggetti testuali. (interessante). Questa tecnica ha il vantaggio di poter disporre di modelli di dati e di sistemi di elaborazione complessi.

% Elementi Milestone: Un elemento milestone è un elemento XML vuoto che segnala un punto monodimensionale in un documento XML. si possono collocare liberamente sintatticamente in un documento XML. Questa strategia è ampiamente utilizzata nella TEI per veicolare le indicazioni sulla messa in pagina di un testo nel documento fonte da cui è stato memorizzato. Gli elementi milestone possono essere usati in coppie virtuali per segnalare i confini di segmenti arbitrari di testo che si sovrappongono agli elementi standard. Insieme a elementi di congiunzione o stand-off markup possono rappresentare virtualmente ogni genere di sovrapposizione, auto-sovrapposizione e segmentazione non contigua. 

%Gli elementi milestone offrono la massima flessibilità sintattica senza costringere a separare markup e contenuto. un parser XML può validare la corretta collocazione di un elemento vuoto rispetto a un modello di contenuto, o verificare che due elementi vuoti siano stati correlati mediante coppie di ID/IDREF. ma non può in alcun modo attribuire funzione strutturale alla sequenza di caratteri contenuta tra due elementi vuoti. non è accessibile come tale a una applicazione XML standard.


% L’ostacolo maggiore consiste nella individuazione di un modello di dati e di un formalismo a esso associato che possa essere validato ed elaborato mediante algoritmi generali e computazionalmente trattabili come avviene per il modello ad albero di XML.
%% rappresentazione di alberi concorrenti in un medesimo documento XML (Xconcur, JIITs)
%% In LMNL la soluzione del problema OH viene trovata uscendo definitivamente fuori dal paradigma gerarchico di XML
%% TexMECS e GODDAD: Markup Languages for Complex Documents. una notazione non molto dissimile da XML (provvede infatti anche strutture come gli attributi) la quale tuttavia permette di esprimere facilmente strutture sovrapposte, auto- sovrapposte e non contigue. In linea teorica questo grafo può esprimere tutte le possibili relazioni tra oggetti testuali linearizzati sottoforma di stringhe di caratteri etichettate mediante markup, inclusi i più complessi casi di auto- sovrapposizione o di frammentazione non contigua e non linearmente ordinata. 



% non possiamo dire apriori che uno schema di codifica testuale coglie l’essenza del testo più e meglio di un altro in base a un qualche assunto metafisico. Ma neppure si può affermare che ogni rappresentazione è vera in quanto costituisce il suo oggetto testo secondo esigenze specifiche e locali.
