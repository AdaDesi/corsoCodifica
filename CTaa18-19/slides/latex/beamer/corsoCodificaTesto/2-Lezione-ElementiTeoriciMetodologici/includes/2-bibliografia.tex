%Jakobson R., Fundamentals of language, The Hague, Mouton & Co., 1956

%Cesare Segre, cap. 5 - Critica testuale, teoria degli insiemi e diasistema, in Semiotica filologica, Torino, Einaudi, 1979, pp. 53-70, ISBN 88-06-08649-9.

%Tito Orlandi

%Ciotti

%Testo e computer

\begin{frame}
    \frametitle{References}
    \addtocounter{nframe}{1}
    \begin{thebibliography}{10}
        \setbeamertemplate{bibliography item}[paper]
        \tiny\bibitem{Ciotti2007} Ciotti, F. (2007). Il testo e l’automa: saggi di teoria e critica computazionale dei testi letterari. Aracne.
        \tiny\bibitem{Lenci2016} Lenci, A., Montemagni S., and Pirrelli V. (2016). Testo e Computer. Elementi Di Linguistica Computazionale. Aulamagna. Carocci.
        \tiny\bibitem{Pierazzo2015} Pierazzo, E. (2015). Digital Scholarly Editing : Theories, Models and Methods. Farnham Surrey: Ashgate.
        \tiny\bibitem{orlandi2010} Orlandi, T. (2010). Informatica testuale: teoria e prassi. Laterza.
        \tiny\bibitem{Pierazzo2016} Driscoll, M. J., and Pierazzo, E. (Eds.). (2016). Digital Scholarly Editing: Theories and Practices (Vol. 4). Open Book Publishers.
        \tiny\bibitem{Segre1985} Segre, C. (1985). Avviamento all'analisi del testo letterario. Einaudi.
        \tiny\bibitem{Segre1979} Segre, C. (1979). Critica testuale, teoria degli insiemi e diasistema, in Semiotica filologica. Einaudi.
    \end{thebibliography}

\end{frame}

\begin{frame}
    \frametitle{References}
    \addtocounter{nframe}{1}
    \begin{thebibliography}{10}
        \setbeamertemplate{bibliography item}[paper]
        \tiny\bibitem{DeRose1990} DeRose, S. J., Durand, D. G., Mylonas, E., and Renear, A. H. (1990). What is text, really? Journal of Computing in Higher Education, 1(2), 3–26.
        \tiny\bibitem{Renear1995} Renear, Allen H.; Mylonas, Elli; Durand, David (1995). Refining our Notion of What Text Really Is: The Problem of Overlapping Hierarchies.
        \tiny\bibitem{ciotti2012} Ciotti F., e Crupi G, a c. di. (2012). Dall’Informatica umanistica alle culture digitali. ROMA : Casa Editrice Università La Sapienza. \href{http://www.editricesapienza.it/node/7688}{open access: http://www.editricesapienza.it/node/7688}
    \end{thebibliography}

\end{frame}

\begin{frame}
    \frametitle{References}
    \addtocounter{nframe}{1}
    \begin{thebibliography}{10}
        
        \setbeamertemplate{bibliography item}[online]
        \tiny\bibitem{TEI Guide} \textit{TEI Guide Reference}. \url{http://www.tei-c.org/}
%        \tiny\bibitem{IBMXML1} IBM XML \textit{Tutorial}, \url{https://www.ibm.com/developerworks/xml/tutorials/xmlintro/xmlintro.html}
%        \tiny\bibitem{w3school} W3School Tutorial \url{https://www.w3schools.com/xml/default.asp}

    \end{thebibliography}

\end{frame}