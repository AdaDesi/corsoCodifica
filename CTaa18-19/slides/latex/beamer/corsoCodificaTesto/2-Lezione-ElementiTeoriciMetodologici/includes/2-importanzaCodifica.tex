% Vedere Slide CHiara. Roberto
% Portabilità e riutilizzabilità
% schema di codifica
% TEI XML focuses on the meaning of text, rather than its appearance.

\begin{frame}
	\frametitle{Importanza della codifica digitale}
	\framesubtitle{Perché effettuare la codifica}
	\addtocounter{nframe}{1}

	\begin{block}{Digitalizzare e codificare un testo}
		Digitalizzare e codificare per poter favorire l'elaborazione e il trattamento automatico dei testi
	\end{block}
\end{frame}

\begin{frame}
	\frametitle{Importanza della codifica digitale}
	\framesubtitle{Perché effettuare la codifica}
	\addtocounter{nframe}{1}

	\begin{block}{Trattamento dei testi}
		\begin{itemize}
			\item  analisi di tipo linguistico (linguistica computazionale,
			      database testuali, corpora linguistics)
			\item analisi di altro tipo (metrica, stilistica, ecc.)
			\item ricerca testuale avanzata
			\item pubblicazione in vari formati (sul Web, come ebook, a
			      stampa)
			\item didattica
		\end{itemize}

	\end{block}
\end{frame}


\begin{frame}
	\frametitle{Importanza della codifica digitale}
	\framesubtitle{Perché effettuare la codifica}
	\addtocounter{nframe}{1}

	\begin{block}{Digitalizzare un testo}
		Per facilitare e garantire una universalità di accesso al loro contenuto
	\end{block}

	\begin{block}{Vantaggi della digitalizzazione}
		\begin{itemize}
			\item Edizioni elettroniche garantiscono diffusione capillare
			      (via Web) e nuove funzionalità (ipertesti, ricerca, ecc.)
			\item Permettono anche di preservare i documenti più antichi
			      (e fragili) riducendone la consultazione diretta
		\end{itemize}
	\end{block}
\end{frame}

\begin{frame}
	\frametitle{Importanza della codifica digitale}
	\framesubtitle{Perché effettuare la codifica}
	\addtocounter{nframe}{1}

	\begin{block}{Superare i problemi dei documenti digitali}
		\begin{itemize}
			\item disponibilità di hardware e software
			\item sistemi proprietari chiusi
			\item elevata obsolescenza e limitata manutenibilità
			\item difficile portabilità su piattaforme diverse
		\end{itemize}
	\end{block}

\end{frame}

\begin{frame}
	\frametitle{Importanza della codifica digitale}
	\framesubtitle{Perché effettuare la codifica}
	\addtocounter{nframe}{1}

	\begin{block}{Sistemi non adatti}
		\begin{itemize}
			\item word processing: WYSIWYG (What You See Is What You Get)
			\item sistemi proprietari chiusi (Word, Adobe, etc)
			\item elevata obsolescenza e limitata manutenibilità (file doc)
			\item difficile portabilità su piattaforme diverse (Windows, Linux)
		\end{itemize}
	\end{block}

\end{frame}

\begin{frame}
	\frametitle{Importanza della codifica digitale}
	\framesubtitle{Perché effettuare la codifica}
	\addtocounter{nframe}{1}

	\begin{block}{Massimizzare le seguenti proprietà: Portabilità}
		\begin{itemize}
			\item indipendenza dall’hardware: processore, supporto, output
			\item indipendenza dal software: sistemi operativi, applicazioni di authoring, applicazioni di visualizzazione
			\item indipendenza dai sistemi di codifica dei caratteri
			\item indipendenza logica: da un particolare processo applicativo
		\end{itemize}
	\end{block}

\end{frame}




%La codifica dell’informazione gode delle seguenti proprietà:
% indipendenza dall’hardware, ovvero da una particolare architettura elaborativa (processore), da un particolare supporto digitale (disco magnetico, disco ottico, etc.), o da un particolare dispositivo o sistema di output (video, stampa);
% indipendenza dal software, sia sistemi operativi, sia applicazioni deputate alla creazione, analisi, manipolazione e visualizzazione di testi elettronici; (formati di dati proprietari mutamente incompatibili)
% indipendenza logica dalle applicazioni ovvero indipendenza semantica dello schema di codifica da un particolare processo applicativo.

% L’archiviazione su supporto digitale del patrimonio letterario e culturale delle culture mondiali deve misurarsi con questi problemi, e adottare degli schemi di codifica capaci di garantire la massima portabilità. 

\begin{frame}
	\frametitle{Importanza della codifica digitale}
	\framesubtitle{Perché effettuare la codifica}
	\addtocounter{nframe}{1}

	\begin{block}{Portabilità}
		Un documento digitale (come qualsiasi risorsa informativa digitale) è portabile se è interscambiabile tra sistemi diversi, riutilizzabile in molteplici processi computazionali anche a distanza di tempo, e integrabile da ulteriori risorse informative omogenee
	\end{block}
\end{frame}

\begin{frame}
	\frametitle{Importanza della codifica digitale}
	\framesubtitle{Perché effettuare la codifica}
	\addtocounter{nframe}{1}

	\begin{block}{Standard}
		Le modalità di codifica devono divenire uno standard per favorire la portabilità del documento digitale.
	\end{block}

	\begin{block}{Standard}
		Insieme di norme relative a una particolare tecnologia emesse da un ente istituzionale nazionale o internazionale deputato a tale scopo (ISO - International Standardization Organization).
	\end{block}
\end{frame}


\begin{frame}
	\frametitle{Importanza della codifica digitale}
	\framesubtitle{Perché effettuare la codifica}
	\addtocounter{nframe}{1}

	\begin{block}{Standard}
		I vantaggi di uno standard formale o informale, oltre alla portabilità sta anche nella sua apertura, ovvero nella disponibilità pubblica delle sue specifiche.
	\end{block}

\end{frame}


\begin{frame}
	\frametitle{Importanza della codifica digitale}
	\framesubtitle{Perché effettuare la codifica}
	\addtocounter{nframe}{1}

	\begin{block}{Standardizzare}

		\begin{itemize}
			\item produrre risultati confrontabili e condivisibili all’interno di
			una comunità che adotta lo stesso tipo di standard
			\item lavorare in modo tale che i prodotti di oggi siano
			usufruibili anche in futuro
			\item codificare testi che possano essere usufruiti anche da chi
			non ha il tipo di piattaforma hardware e software che ha
			prodotto la codifica
		\end{itemize}
		 
	\end{block}

\end{frame}


\begin{frame}
	\frametitle{Importanza della codifica digitale}
	\framesubtitle{Perché effettuare la codifica}
	\addtocounter{nframe}{1}

	\begin{block}{Linguaggio di codifica}
		Se per predisporre una codifica del testo utilizziamo un linguaggio dotato di una sintassi che permetta di specificare le relazioni tra gli elementi, essa può essere usata per rappresentare la struttura e l’organizzazione del testo a un determinato livello di descrizione, o i rapporti tra elementi appartenenti a diversi livelli.
	\end{block}

\end{frame}

\begin{frame}
	\frametitle{Importanza della codifica digitale}
	\framesubtitle{Perché effettuare la codifica}
	\addtocounter{nframe}{1}

	\begin{block}{Linguaggio di codifica del contenuto}
		Il sistema di codifica deve assistere nel processo di rappresentazione del testo focalizzando l’attenzione sul contenuto (o sulla struttura del contenuto) piuttosto che sulla sua forma grafica.
	\end{block}

	\begin{block}{Linguaggio di codifica dichiarativo}
		I sistemi di codifica dichiarativa si prestano ottimamente per rappresentare strutture complesse come riferimenti incrociati e collegamenti tra elementi all’interno di un testo, ma anche tra più testi.
	\end{block}

\end{frame}

\begin{frame}
	\frametitle{Importanza della codifica digitale}
	\framesubtitle{Perché effettuare la codifica}
	\addtocounter{nframe}{1}

	\begin{block}{Linguaggio di codifica dichiarativo - Markup}
		I sistemi di markup dichiarativo introducono consistenti vantaggi poiché un medesimo schema di codifica dichiarativo che corrisponde ad un modello di codifica può essere utilizzato in molteplici forme di trattamento informatico.
	\end{block}

	\textit{markup dichiarativo}: annotano la struttura e il significato degli elementi costitutivi del testo, tralasciandone l’aspetto.

\end{frame}

\begin{frame}
	\frametitle{Importanza della codifica digitale}
	\framesubtitle{Perché effettuare la codifica}
	\addtocounter{nframe}{1}

	\begin{block}{Descrizione formale del testo}
		Così è possibile descrivere formalmente le caratteristiche di un testo in modo indipendente da particolari finalità di trattamento da contingenti forme di presentazione grafica su un qualsivoglia supporto fisico
	\end{block}

\end{frame}


\begin{frame}
	\frametitle{Importanza della codifica digitale}
	\framesubtitle{Perché effettuare la codifica}
	\addtocounter{nframe}{1}

	\begin{block}{Linguaggio di codifica come linguaggio teorico}
		In questo modo la codifica permette allo studioso di esplicitare le sue ipotesi interpretative.
	\end{block}

	\begin{block}{Linguaggio di codifica dichiarativo flessibile}
		I linguaggi di markup dichiarativi permettono di predicare l’appartenenza di un dato segmento testuale a una classe di strutture testuali definita dall’utente.
	\end{block}

\end{frame}

\begin{frame}
	\frametitle{Importanza della codifica digitale}
	\framesubtitle{Perché effettuare la codifica}
	\addtocounter{nframe}{1}

	\begin{block}{Gerarchia ordinata di oggetti di contenuto (OHCO)}
		I linguaggi di markup dichiarativi sono particolarmente efficienti dal punto di vista computazionale se modellano una struttura gerarchica. 
	\end{block}

	\begin{block}{Linguaggi di codifica gerarchici}
		La sintassi del linguaggio di codifica gerarchico può essere usata per rappresentare le relazioni tra gli elementi strutturali di un testo, a un determinato livello di descrizione.
	\end{block}

\end{frame}

\begin{frame}
	\frametitle{Importanza della codifica digitale}
	\framesubtitle{Perché effettuare la codifica}
	\addtocounter{nframe}{1}

	\begin{block}{Gerarchia ordinata di oggetti di contenuto (OHCO)}
	 I linguaggi di markup dichiarativi gerarchici, in particolare SGML prima e XML poi, si sono rivelati ottimi strumenti di supporto alla codifica e all’analisi computazionale dei testi.
	\end{block}

\end{frame}


% La disposizione alla rappresentazione di strutture astratte non pone limiti alla natura e tipologia delle caratteristiche testuali che si possono codificare in un testo elettronico. Queste possono essere utilizzate indifferentemente 

% Infatti un database offre dei notevoli vantaggi dal punto di vista delle prestazioni computazionali e della velocità di ricerca, anche se richiede in generale una ingente quantità di memoria per l’archiviazione.

